%%=============================================================================
%% Methodologie
%%=============================================================================

\chapter{\IfLanguageName{dutch}{Methodologie}{Methodology}}
\label{ch:methodologie}

%% TODO: Hoe ben je te werk gegaan? Verdeel je onderzoek in grote fasen, en
%% licht in elke fase toe welke stappen je gevolgd hebt. Verantwoord waarom je
%% op deze manier te werk gegaan bent. Je moet kunnen aantonen dat je de best
%% mogelijke manier toegepast hebt om een antwoord te vinden op de
%% onderzoeksvraag.

Om te kunnen onderzoeken welke verschillende narrow ai's men dient te combineren om een impressie te wekken dat men met een general ai in contact staat werd er gefocust op 2 belangrijke aspecten:

\begin{itemize}
    \item De menselijke aspecten van een Artificial General Intelligence
    \item Welke narrow ai's dient men te combineren om deze menselijke aspecten na te bootsten
\end{itemize}

Na dit theoretisch te benaderen aan de hand van een onderzoek, werd er ook een enquete gehouden bij 3 verschillende doelgroepen om te kunnen onderzoeken waar voor de meeste mensen de AI aan dient te voldoen om een gevoel te kunnen nabootsen die voldoet aan de menselijke aspecten van een artificial general intelligence.

\newpage

\section{Menselijke aspecten Artificial General Intelligence}

Uiteraard is het schetsen van alle eigenschappen van een mens onbegonnen werk, dit is zowel bij elke persoon anders omdat iedereen andere karaktereigenschappen heeft. Ook zijn zaken zoals bijvoorbeeld een geweten of moraal één van de moeilijkste zaken om in een technologie te gieten. 

In dit onderzoek werd er op zoek gegaan naar de meest voor de hand liggende eigenschappen van een mens, niet bepaald het emotionele gegeven, dit kan misschien wel een leuke uitbreiding zijn op deze bachelorproef voor een bachelorproef in het psychologie veld.

Om dit onderzoek te schetsen ging men er van uit dat men een ai zou willen opzetten die via bijvoorbeeld een videocall met video en spraak contact zou hebben met een persoon, en dat de tegenpartij niet kan onderscheiden of er contact gemaakt wordt met een technologie of een echte persoon.

Bovenstaande case werd opgesplitst in de drie meest voor de hand liggende delen:

\begin{itemize}
    \item Beeld
    \item Spraak
    \item Kennis
\end{itemize}

In de volgende subsecties werd dit meer in detail onderzocht

\newpage

\section{Beeld}

Op vlak van beeld werd er van uit gegaan dat onderstaande zaken het belangrijkste waren:

\begin{itemize}
    \item Gezichtsherkenning
    \item Emotieherkenning
    \item Generatief antagonistennetwerk (deepfakes)
\end{itemize}

Gezichtsherkenning: Een AI die in staat is om gezichten te herkennen en te onderscheiden. Dit kan helpen om gepersonaliseerde en aangepaste conversaties te bieden, bijvoorbeeld door de herkenning van emoties op het gezicht van de gesprekspartner.

Emotieherkenning: Het vermogen om emoties te herkennen en te interpreteren is essentieel voor effectieve communicatie tussen mensen. Een AI die menselijke emoties kan herkennen en daarop kan reageren, kan helpen om een meer natuurlijke conversatie te creëren.

Deepfakes: Het gebruik van generatieve antagonistennetwerken, zoals deepfakes, kan helpen om een meer realistische en overtuigende conversaties te creëren. Door het gebruik van deepfakes, kan de AI gezichtsuitdrukkingen en andere non-verbale signalen nabootsen, waardoor de conversatie nog overtuigender wordt.

\newpage

\section{Spraak}

Op vlak van spraak werd er van uit gegaan dat onderstaande zaken het belangrijkste waren:

\begin{itemize}
    \item Spraakherkenning
    \item Spraakverwerking
    \item Taalherkenning
    \item Taalverwerking
    \item Synthetische stemtechnologie
\end{itemize}

Spraakherkenning: Een AI die in staat is om spraak te herkennen kan helpen om een meer natuurlijke te creëren. Door de AI in staat te stellen om spraak te herkennen, kan het beter reageren op de gesproken vragen en opmerkingen van de tegenpartij.

Spraakverwerking: Dit gaat om een AI die in staat is om grammatica en woordbetekenissen te begrijpen. Ook hoort het begrijpen van context en intentie hier bij. Dit spreekt voor zich dat dit van groot belang is voor het creëren van een communicatie die natuurlijk aanvoelt.

Taalherkenning: De mogelijkheid om taal te herkennen is essentieel voor het ontwikkelen van een AI die menselijke communicatie kan nabootsen. Door de AI in staat te stellen taal te herkennen, kan het beter begrijpen wat de tegenpartij zegt en vervolgens de juiste reacties geven.

Taalverwerking: Taalverwerking is de volgende stap na taalherkenning. Het is het vermogen van de AI om de taal van de gesprekspartner te interpreteren en er op te reageren. Een effectieve taalverwerking is van enorm belang om een natuurlijke en intuïtieve conversatie te creëren.

Synthetische stemtechnologie: Door synthetische stemtechnologie, eigenlijk een onderdeel van spraakverwerking, te gebruiken, kan de AI menselijke spraak nabootsen, waardoor de conversatie overtuigender en realistischer wordt. Dit slaat voornamelijk op menselijk klinken. Het gebruik van synthetische stemtechnologie is daarom ook zeer belangrijk.

\newpage

\section{Kennis}

Op vlak van kennis werd er van uit gegaan dat onderstaande zaken het belangrijkste waren:

\begin{itemize}
    \item Algemene kennis en skills verwerven en verbeteren
    \item Persoonlijkheidsmodellering
    \item Contextueel begrip
\end{itemize}

Kennis en vaardigheden: Het verwerven en verbeteren van kennis en vaardigheden is essentieel voor het ontwikkelen van een AI die menselijke communicatie kan nabootsen. Door de AI te programmeren met uitgebreide kennis en vaardigheden, kan het beter reageren op de vragen en opmerkingen van de gesprekspartner en kan het een meer overtuigende communicatie tot stand brengen.

Persoonlijkheidsmodellering: Een belangrijk aspect van effectieve communicatie is dat de gesprekspartner de persoonlijkheid van de ander begrijpt. Een AI die is geprogrammeerd om de persoonlijkheid van de gesprekspartner te modelleren, kan helpen om een meer gepersonaliseerde en aangepaste conversatie te bieden.

Contextueel begrip: Mensen gebruiken vaak impliciete contextuele aanwijzingen om de betekenis van gesproken taal te begrijpen. Een AI die in staat is om de context van een gesprek te begrijpen en daarop te reageren. Dit is tevens een onderdeel van spraakverwerking, maar blijft ook op vlak van kennis zeer belangrijk.

\newpage