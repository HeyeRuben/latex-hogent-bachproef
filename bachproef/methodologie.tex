%%=============================================================================
%% Methodologie
%%=============================================================================

\chapter{\IfLanguageName{dutch}{Methodologie}{Methodology}}
\label{ch:methodologie}

%% TODO: Hoe ben je te werk gegaan? Verdeel je onderzoek in grote fasen, en
%% licht in elke fase toe welke stappen je gevolgd hebt. Verantwoord waarom je
%% op deze manier te werk gegaan bent. Je moet kunnen aantonen dat je de best
%% mogelijke manier toegepast hebt om een antwoord te vinden op de
%% onderzoeksvraag.

Om te kunnen onderzoeken welke verschillende narrow ai's men dient te combineren om een impressie te wekken dat men met een general ai in contact staat werd er gefocust op 2 belangrijke aspecten:

\begin{itemize}
    \item De menselijke aspecten van een Artificial General Intelligence
    \item Welke narrow ai's dient men te combineren om deze menselijke aspecten na te bootsten
\end{itemize}

Na dit theoretisch te benaderen aan de hand van een onderzoek, werd er ook een enquete gehouden bij 3 verschillende doelgroepen om te kunnen onderzoeken waar voor de meeste mensen de AI aan dient te voldoen om een gevoel te kunnen nabootsen die voldoet aan de menselijke aspecten van een artificial general intelligence.

\newpage

\section{Menselijke aspecten Artificial General Intelligence}

Uiteraard is het schetsen van alle eigenschappen van een mens onbegonnen werk, dit is bij elke persoon anders omdat iedereen andere karaktereigenschappen heeft. Ook zijn zaken zoals bijvoorbeeld een geweten of moraal één van de moeilijkste zaken om in een technologie te gieten. 

In dit onderzoek werd er op zoek gegaan naar de meest voor de hand liggende eigenschappen van een mens, niet bepaald het emotionele gegeven, dit kan misschien wel een leuke uitbreiding zijn op deze bachelorproef voor een bachelorproef in het psychologie veld.

Om dit onderzoek te schetsen ging men er van uit dat men een ai zou willen opzetten die via bijvoorbeeld een videocall met video en spraak contact zou hebben met een persoon, en dat de tegenpartij niet kan onderscheiden of er contact gemaakt wordt met een technologie of een echte persoon.

Bovenstaande case werd opgesplitst in de drie meest voor de hand liggende delen:

\begin{itemize}
    \item Beeld
    \item Spraak
    \item Kennis
\end{itemize}

In de volgende subsecties werd dit meer in detail onderzocht

\newpage

\section{Beeld}

Op vlak van beeld werd er van uit gegaan dat onderstaande zaken het belangrijkste waren:

\begin{itemize}
    \item Gezichtsherkenning
    \item Emotieherkenning
    \item Deepfakes
\end{itemize}

Gezichtsherkenning: Een AI die in staat is om gezichten te herkennen en te onderscheiden. Dit kan helpen om gepersonaliseerde en aangepaste conversaties te bieden, bijvoorbeeld door de herkenning van emoties op het gezicht van de gesprekspartner.

Emotieherkenning: Het vermogen om emoties te herkennen via het gelaat en te interpreteren is essentieel voor effectieve communicatie tussen mensen. Een AI die menselijke emoties kan herkennen en daarop kan reageren, kan helpen om een meer natuurlijke conversatie te creëren.

Deepfakes: Het gebruik van generatieve antagonistennetwerken, zoals deepfakes, kan helpen om een meer realistische en overtuigende conversaties te creëren. Door het gebruik van deepfakes, kan de AI gezichtsuitdrukkingen en andere non-verbale signalen nabootsen, waardoor de conversatie nog overtuigender wordt.

\newpage

\subsection{Gezichtsherkenning}

Gezichtsherkenning is een technologie die het mogelijk maakt om individuele gezichten te identificeren en te verifiëren op basis van specifieke kenmerken, zoals de afstand tussen de ogen, de neus en de mond, de vorm van het gezicht en de verhoudingen tussen deze kenmerken.

Er zijn verschillende methoden voor gezichtsherkenning, waaronder 2D-gezichtsherkenning, 3D-gezichtsherkenning en gezichtsherkenningsalgoritmen gebaseerd op artificiele intelligentie (AI). 2D-gezichtsherkenning maakt gebruik van platte afbeeldingen van het gezicht om het te herkennen, terwijl 3D-gezichtsherkenning gebruik maakt van driedimensionale modellen van het gezicht om meer gedetailleerde en nauwkeurige identificatie te bieden.

Zoals hierboven besproken is dit interessant voor deze case om via deze manier gepersonaliseerde en aangepaste conversaties te bieden. Wanneer men focust op gezichtsherkenning gebasseerd op artificiele intelligentie dan komt men op  artificiele intelligentie (AI) die gebruik maakt van deep learning-technieken, waaronder convolutionele neurale netwerken (CNN's) en recurrente neurale netwerken (RNN's), om gezichten te herkennen en te identificeren.

\subsubsection{Onderzoek naar combinatie van Artificial Narrow Intelligences}

Wanneer men verschillende artificial narrow intelligences zou gaan willen combineren om gezichtsherkenning goed te laten verlopen denkt men in eerste instantie aan:

\begin{itemize}
    \item Beeldverwerking
    \item Patroonherkenning
    \item Machine learning
\end{itemize}

In eerste instantie is het belangrijk dat dit onderdeel van de software in staat is om digitale beelden van gezichten te analyseren en verwerken. Via deze manier kan men belangrijke kenmerken van een gezicht, zoals bijvoorbeeld de ogen, neus en mond te identificeren. Dit is onderdeel van beeldverwerking.

Vervolgens zou men via patroonherkenning moeten kunnen herkennen welke combinaties van kenmerken bij een bepaald gezicht horen. Via deze gekende patronen kan men de identiteit van een persoon bepalen.

Bij zowel beeldverwerking als patroonherkenning is machine learning een belangrijke tool om het gezichtsherkenningssysteem te verbeteren en te optimaliseren.

Zo kan machine learning bij beeldverwerking worden gebruikt om het systeem te trainen om automatisch bepaalde kenmerken van een gezicht te herkennen, zoals de positie van de ogen, neus en mond. Hier kunnen convolutionele neurale netwerken voor worden gebruikt, die zeer handig zijn voor beeldverwerking.

Ook voor patroonherkenning kan machine learning worden gebruikt om het systeem te trainen om automatisch patronen in gezichtskenmerken te herkennen die uniek zijn voor een bepaald persoon.

\subsection{Emotieherkenning}

Emotieherkenning is het proces waarbij software of technologie wordt gebruikt om menselijke emoties te detecteren en te interpreteren op basis van gezichtsuitdrukkingen, stemintonatie en andere fysieke signalen.

Het maakt gebruik van verschillende methoden, waaronder beeld- en spraakverwerking, machine learning en artificiele intelligentie om emoties te identificeren en te classificeren. In het geval van gezichtsherkenning wordt gebruik gemaakt van computeralgoritmen om gezichtsuitdrukkingen te analyseren en te interpreteren, zoals glimlachen, fronsen en het bewegen van de wenkbrauwen.
 
Bij stemherkenning worden signalen van de stem opgevangen en geanalyseerd om tonen, verbuigingen van woorden en andere stemkenmerken te detecteren die indicatief zijn voor verschillende emoties, hier komt men in het onderdeel spraak nog op terug.

Binnen emotieherkenning wordt artificiele intelligentie (AI) gebruikt om complexe patronen in de gegevens te ontdekken en te interpreteren. AI-algoritmen kunnen worden getraind om gezichtsuitdrukkingen en stemkenmerken te herkennen die specifiek zijn voor bepaalde emoties.

De meest gebruikte methode binnen AI voor emotieherkenning is machine learning. Hierbij worden de algoritmen getraind op een grote hoeveelheid gegevens om patronen te ontdekken en voorspellingen te doen over nieuwe gegevens. Het trainen van de algoritmen gebeurt meestal met behulp van datasets die handmatig zijn gelabeld door menselijke experts. Deze labels worden gebruikt om het algoritme te leren welke kenmerken overeenkomen met welke emoties.

Naast machine learning zijn er ook andere AI-technieken die worden gebruikt in emotieherkenning, zoals deep learning en neurale netwerken. Deze methoden maken gebruik van complexe wiskundige modellen die zijn ontworpen om grote hoeveelheden gegevens te verwerken en complexe patronen te identificeren.

\subsubsection{Onderzoek naar combinatie van Artificial Narrow Intelligences}

Wanneer men verschillende artificial narrow intelligences zou gaan willen combineren om emotieherkenning goed te laten verlopen denkt men in eerste instantie aan:

\begin{itemize}
    \item Beeldverwerking
    \item Patroonherkenning
    \item Machine learning
\end{itemize}

De benodigde articial narrow intelligences die men dient te combineren zijn in eerste instantie dezelfde als bij gezichtsherkenning, omdat ze beide steunen op het visuele aspect.

Zo zal beeldverwerking dienen om digitale beelden van gezichten te analyseren en verwerken om zo kenmerken in verband met gezichtsuitdrukkingen, zoals oog-, mond- en wenkbrauwpositie te identificeren.

Patroonherkenning zal dan op zijn beurt moeten kunnen herkennen welke patronen in gezichtsuitdrukkingen bij bepaalde emoties horen en deze patronen kunnen identificeren om op deze manier de emotie van een persoon te achterhalen.

Machine learning zorgt er op zijn beurt dan weer voor dat men de AI kan trainen via voorbeelden van emoties in gezichtsuitdrukkingen, om zo emoties nauwkeuriger te herkennen en te classificeren.

Dankzij deze combinatie kan een emotieherkenningsysteem worden gemaakt dat in staat is om emoties nauwkeurig te identificeren en te classificeren op basis van gezichtsuitdrukkingen.

\subsection{Deepfake}

Een deepfake is een type vervalste video, afbeelding of audio waarbij artificiele intelligentie (AI) wordt gebruikt om de inhoud te manipuleren en te creëren. De term 'deepfake' komt van de combinatie van 'deep learning' en 'fake'.

Deepfakes worden gemaakt met behulp van machine learning-algoritmen, zoals generatieve neurale netwerken, om een ​​realistische, vervalste weergave te creëren van iemand die niet echt bestaat, of om een bestaand persoon te manipuleren om iets te zeggen of te doen dat hij of zij nooit heeft gedaan.

Een van de meest voorkomende toepassingen van deepfakes is om de gezichten van bekende personen te vervangen door de gezichten van andere mensen in video's. Dit is dus ideaal voor deze case. Deepfakes worden vaak gecombineerd met andere machine learning-technieken zoals gezichts- en stemherkenning.

\subsubsection{Onderzoek naar combinatie van Artificial Narrow Intelligences}

Wanneer men verschillende artificial narrow intelligences zou gaan willen combineren om een deepfake weer te geven denkt men in eerste instantie aan:

\begin{itemize}
    \item Generatief Antagonisten Netwerk
\end{itemize}

Voor het tonen van een beeld dat gegenereerd is en op een echt persoon moet lijken denkt men voornamelijk aan deepfakes. Wanneer men bij een deepfake focust op enkel het beeld dan kan een deepfake getrained worden via een Generatief Antagonisten Netwerk.

Deze AI kan worden gebruikt om nieuwe en realistische beelden te genereren. Via deze weg kan men realistische deepfakes creëren waar het voor de gebruiker van de AI niet merkbaar is dat deze door een AI gegenereerd is.

\section{Spraak}

Op vlak van spraak werd er van uit gegaan dat onderstaande zaken het belangrijkste waren:

\begin{itemize}
    \item Spraakherkenning
    \item Spraakverwerking
    \item Taalherkenning
    \item Taalverwerking
    \item Synthetische stemtechnologie
\end{itemize}

Spraakherkenning: Een AI die in staat is om spraak te herkennen kan helpen om een meer natuurlijke te creëren. Door de AI in staat te stellen om spraak te herkennen, kan het beter reageren op de gesproken vragen en opmerkingen van de tegenpartij.

Spraakverwerking: Dit gaat om een AI die in staat is om grammatica en woordbetekenissen te begrijpen. Ook hoort het begrijpen van context en intentie hier bij. Dit spreekt voor zich dat dit van groot belang is voor het creëren van een communicatie die natuurlijk aanvoelt.

Taalherkenning: De mogelijkheid om taal te herkennen is essentieel voor het ontwikkelen van een AI die menselijke communicatie kan nabootsen. Door de AI in staat te stellen taal te herkennen, kan het beter begrijpen wat de tegenpartij zegt en vervolgens de juiste reacties geven.

Taalverwerking: Taalverwerking is de volgende stap na taalherkenning. Het is het vermogen van de AI om de taal van de gesprekspartner te interpreteren en er op te reageren. Een effectieve taalverwerking is van enorm belang om een natuurlijke en intuïtieve conversatie te creëren.

Synthetische stemtechnologie: Door synthetische stemtechnologie, eigenlijk een onderdeel van spraakverwerking, te gebruiken, kan de AI menselijke spraak nabootsen, waardoor de conversatie overtuigender en realistischer wordt. Dit slaat voornamelijk op menselijk klinken. Het gebruik van synthetische stemtechnologie is daarom ook zeer belangrijk.

\newpage

\subsection{Spraakherkenning}

Spraakherkenning, is een technologie waarmee computers menselijke spraak kunnen begrijpen en interpreteren. Het proces van spraakherkenning omvat het opnemen van spraakgeluiden, deze omzetten in elektrische signalen en vervolgens de omzetting van deze elektrische signalen in digitale gegevens die kunnen worden begrepen en geanalyseerd door de computer.

Er zijn twee soorten spraakherkenning: onafhankelijke spraakherkenning en spraakherkenning met grammatica. Onafhankelijke spraakherkenning is de eenvoudigste vorm, waarbij de computer naar specifieke woorden luistert en deze herkent. Spraakherkenning met grammatica is complexer en wordt gebruikt om complete zinnen te herkennen. Het vereist een taalmodel dat bekend is met de grammatica en de vocabulaire van de gesproken taal.

\subsection{Spraakverwerking}

Spraakverwerking is een technologie die de computer in staat stelt om natuurlijke spraak te begrijpen, te genereren en te transformeren. Het omvat verschillende processen, zoals spraakherkenning, spraaksynthese, spraakvertaling en spraakanalyse.

Spraakverwerking begint met spraakopname, waarbij de stem van een spreker wordt omgezet in digitale signalen door middel van microfoons en signaalverwerkingstechnieken. Deze digitale signalen worden vervolgens geanalyseerd en verwerkt door de computer om de spraakinformatie te extraheren en te begrijpen.

Spraakherkenning is een van de belangrijkste toepassingen van spraakverwerking. Het omvat het omzetten van gesproken woorden en zinnen in tekstuele vorm, wat nuttig is voor spraakgestuurde systemen, zoals virtuele assistenten, spraak-naar-tekst transcriberen, etc.

Spraaksynthese, daarentegen, is het proces waarbij de computer natuurlijk klinkende spraak genereert op basis van tekst of geschreven informatie. Deze technologie wordt veel gebruikt in audioboeken, spraakgestuurde navigatie en andere toepassingen waarbij gesproken informatie nodig is. Hier komt men later op terug.

Spraakvertaling is een andere belangrijke toepassing van spraakverwerking, waarbij de computer gesproken woorden en zinnen in verschillende talen kan vertalen. Dit kan bijvoorbeeld worden gebruikt bij het reizen of het communiceren met mensen die een andere taal spreken.

Spraakverwerking maakt gebruik van verschillende geavanceerde technologieën, zoals machine learning, deep learning en neurale netwerken. Deze technologieën hebben spraakverwerking nauwkeuriger en efficiënter gemaakt, waardoor deze technologie breed kan worden toegepast in diverse domeinen en toepassingen.

\subsection{Taalherkenning}

Taalherkenning is een technologie die computers in staat stelt om natuurlijke taal te begrijpen en te verwerken. Het omvat verschillende processen, zoals taalherkenning, taalgeneratie, taalvertaling en taalanalyse.

Taalherkenning begint met het identificeren en begrijpen van de structuur van de zin, inclusief de grammaticale en semantische betekenis van de woorden. Dit omvat het identificeren van de onderwerpen, werkwoorden, objecten en andere belangrijke elementen van de zin. Dit proces is afhankelijk van de beschikbaarheid van taalkundige regels, woordenboeken en taalmodellen.

Taalgeneratie daarentegen is het proces waarbij computers tekstuele informatie omzetten in natuurlijke taal. Dit kan bijvoorbeeld worden gebruikt bij chatbots en virtuele assistenten, waarbij de computer de gebruiker kan antwoorden in menselijke taal, zoals ook in deze concrete case het geval zou moeten zijn.

Taalherkenning en taalgeneratie maken gebruik van verschillende technologieën, zoals machine learning, deep learning en natuurlijke taalverwerking. Deze technologieën maken gebruik van complexe algoritmes en modellen die taal kunnen begrijpen en verwerken, waardoor computers steeds beter worden in het begrijpen en genereren van natuurlijke taal.

Taalvertaling is een andere belangrijke toepassing van taalherkenning en maakt het mogelijk om tekstuele informatie van de ene taal naar de andere te vertalen.

Taalanalyse is het proces waarbij computers de inhoud van teksten analyseren om inzicht te krijgen in verschillende aspecten, zoals onderwerp, sentiment, intentie en entiteiten. Dit kan bijvoorbeeld worden gebruikt bij het analyseren van sociale media-berichten of klantrecensies.

Taalherkenning wordt veel gebruikt in verschillende domeinen en toepassingen, zoals virtuele assistenten, chatbots, spraak-naar-tekst transcripties, sentimentanalyse, etc.

\subsection{Taalverwerking}

Taalverwerking is het geautomatiseerde proces waar bovenstaande zaken samen komen. Tijdens dit proces proberen computers natuurlijke taal te begrijpen en verwerken. Dit proces omvat verschillende taken, zoals taalherkenning, taalgeneratie, taalvertaling, taalanalyse en sentimentanalyse.

Taalverwerking maakt gebruik van verschillende technologieën, zoals machine learning, deep learning en natuurlijke taalverwerking (NLP). Deze technologieën maken gebruik van complexe algoritmes en modellen die taal kunnen begrijpen en verwerken, waardoor computers steeds beter worden in het begrijpen en produceren van natuurlijke taal. 

\subsection{Synthetische stemtechnologie}

Synthetische stemtechnologie (ook wel text-to-speech of TTS genoemd) is een vorm van artificiele intelligentie (AI) die computers in staat stelt om natuurlijke spraak te produceren uit tekstuele input. Het proces begint met de invoer van tekstuele informatie, waarna het TTS-algoritme deze informatie analyseert en omzet in fonemen, die de kleinste geluidseenheden van de taal zijn. Vervolgens worden deze fonemen gecombineerd en uitgesproken als spraakgeluiden om een natuurlijke spraak te creëren.

Er zijn verschillende soorten synthetische stemtechnologieën, deze varieeren van eenvoudige systemen die standaardstemmen gebruiken tot meer geavanceerde systemen die gebruikmaken van deep learning-algoritmes om menselijke spraak te simuleren en een breder scala aan stemvariaties te bieden. Bovendien kunnen sommige systemen worden getraind op specifieke stemmen of accenten om een nog natuurlijkere spraak te produceren.

Synthetische stemtechnologie wordt gebruikt in verschillende toepassingen, zoals spraakgestuurde assistenten, navigatie-apps, e-learning-toepassingen, etc. Het biedt voordelen zoals een efficiëntere communicatie.

\section{Kennis}

Op vlak van kennis werd er van uit gegaan dat onderstaande zaken het belangrijkste waren:

\begin{itemize}
    \item Algemene kennis en skills verwerven en verbeteren
    \item Persoonlijkheidsmodellering
    \item Contextueel begrip
\end{itemize}

Kennis en vaardigheden: Het verwerven en verbeteren van kennis en vaardigheden is essentieel voor het ontwikkelen van een AI die menselijke communicatie kan nabootsen. Door de AI te programmeren met uitgebreide kennis en vaardigheden, kan het beter reageren op de vragen en opmerkingen van de gesprekspartner en kan het een meer overtuigende communicatie tot stand brengen.

Persoonlijkheidsmodellering: Een belangrijk aspect van effectieve communicatie is dat de gesprekspartner de persoonlijkheid van de ander begrijpt. Een AI die is geprogrammeerd om de persoonlijkheid van de gesprekspartner te modelleren, kan helpen om een meer gepersonaliseerde en aangepaste conversatie te bieden.

Contextueel begrip: Mensen gebruiken vaak impliciete contextuele aanwijzingen om de betekenis van gesproken taal te begrijpen. Een AI die in staat is om de context van een gesprek te begrijpen en daarop te reageren. Dit is tevens een onderdeel van spraakverwerking, maar blijft ook op vlak van kennis zeer belangrijk.

\newpage

\subsection{Algemene kennis en skills verwerven en verbeteren}

Een AI die zijn algemene kennis en vaardigheden kan verbeteren, wordt vaak aangeduid als een 'zelflerend' of 'zelfverbeterend' AI-systeem. Dit type AI maakt gebruik van machine learning-technieken, zoals deep learning en reinforcement learning, om zelfstandig te leren en te groeien in kennis en vaardigheden.

Om een echt gesprek met een mens te kunnen hebben, zou deze AI in staat moeten zijn om te communiceren met de gebruiker, zowel via spraak als tekst, en om verschillende taken uit te voeren. Het AI-systeem zou moeten kunnen begrijpen wat de gebruiker zegt en de intentie achter de woorden begrijpen, om zo de beste actie te bepalen die moet worden ondernomen. Het zou ook kennis moeten hebben van verschillende onderwerpen, zoals nieuws, weer, sport, etc, om op verzoek relevante informatie te kunnen geven.

Om zijn kennis en vaardigheden te verbeteren, zou het AI-systeem gebruik kunnen maken van verschillende technieken, zoals het bijhouden van logboeken van eerdere interacties met de gebruiker en het analyseren van deze gegevens om het systeem te verbeteren. Ook kan het systeem leren van nieuwe informatiebronnen, zoals het lezen van online artikelen en het bekijken van video's om zijn kennis te vergroten.

\subsection{Persoonlijkheidsmodellering}

Persoonlijkheidsmodellering is het proces van het analyseren en voorspellen van menselijke persoonlijkheden op basis van verzamelde gegevens. Dit proces wordt vaak ondersteund door artificiele intelligentie (AI) en machine learning-algoritmen.

Om een persoonlijkheidsmodel te maken, worden gegevens verzameld over iemands gedrag, interesses, attitudes, waarden en andere persoonlijke kenmerken. Deze gegevens worden vervolgens geanalyseerd en georganiseerd in verschillende dimensies van persoonlijkheid, zoals extraversie, vriendelijkheid, openheid voor ervaringen, emotionele stabiliteit en consciëntieusheid.

Op basis van deze dimensies en de bijbehorende gegevens kunnen AI-algoritmen patronen en verbanden identificeren die kunnen worden gebruikt om persoonlijkheidsprofielen op te stellen en toekomstig gedrag te voorspellen. Deze profielen kunnen vervolgens worden gebruikt om geautomatiseerde persoonlijke assistenten en chatbots te creëren die zich kunnen aanpassen aan de persoonlijkheid van een gebruiker en betere, meer gepersonaliseerde interacties te kunnen bieden. Ideaal voor deze case dus.

\subsection{Contextueel begrip}

Contextueel begrip verwijst naar het vermogen van een computerprogramma of AI-systeem om de context te begrijpen waarin taal of informatie wordt gebruikt en om de betekenis ervan te interpreteren op basis van deze context.

Contextueel begrip wordt meestal bereikt door gebruik te maken van natuurlijke taalverwerking en machine learning-technieken om woorden en zinnen te analyseren en te vergelijken met andere voorbeelden van vergelijkbare contexten. Door deze vergelijkingen kan een AI-systeem de betekenis van woorden en zinnen beter begrijpen en contextueel relevante antwoorden en acties produceren. Dit vermogen is van cruciaal belang voor chatbots, spraakgestuurde assistenten en andere AI-toepassingen waarbij de computer moet begrijpen wat de gebruiker bedoelt, zelfs als de gebruiker niet altijd dezelfde woorden gebruikt.