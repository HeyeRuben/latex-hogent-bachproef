%===============================================================================
% LaTeX sjabloon voor de bachelorproef toegepaste informatica aan HOGENT
% Meer info op https://github.com/HoGentTIN/bachproef-latex-sjabloon
%===============================================================================

\documentclass{bachproef-tin}

\usepackage{hogent-thesis-titlepage} % Titelpagina conform aan HOGENT huisstijl

%%---------- Documenteigenschappen ---------------------------------------------
% TODO: Vul dit aan met je eigen info:

% De titel van het rapport/bachelorproef
\title{Het creeren van een menselijke conversatie-ervaring met behulp van combinaties van narrow AI-technologieën}

% Je eigen naam
\author{Ruben Heye}

% De naam van je promotor (lector van de opleiding)
\promotor{Geert Van Boven}

% De naam van je co-promotor. Als je promotor ook je opdrachtgever is en je
% dus ook inhoudelijk begeleidt (en enkel dan!), mag je dit leeg laten.
\copromotor{Robin Menschaert}

% Indien je bachelorproef in opdracht van/in samenwerking met een bedrijf of
% externe organisatie geschreven is, geef je hier de naam. Zoniet laat je dit
% zoals het is.
\instelling{---}

% Academiejaar
\academiejaar{2022-2023}

% Examenperiode
%  - 1e semester = 1e examenperiode => 1
%  - 2e semester = 2e examenperiode => 2
%  - tweede zit  = 3e examenperiode => 3
\examenperiode{2}

%===============================================================================
% Inhoud document
%===============================================================================

\begin{document}

%---------- Taalselectie -------------------------------------------------------
% Als je je bachelorproef in het Engels schrijft, haal dan onderstaande regel
% uit commentaar. Let op: de tekst op de voorkaft blijft in het Nederlands, en
% dat is ook de bedoeling!

%\selectlanguage{english}

%---------- Titelblad ----------------------------------------------------------
\inserttitlepage

%---------- Samenvatting, voorwoord --------------------------------------------
\usechapterimagefalse
%%=============================================================================
%% Voorwoord
%%=============================================================================

\chapter*{\IfLanguageName{dutch}{Woord vooraf}{Preface}}
\label{ch:voorwoord}

%% TODO:
%% Het voorwoord is het enige deel van de bachelorproef waar je vanuit je
%% eigen standpunt (``ik-vorm'') mag schrijven. Je kan hier bv. motiveren
%% waarom jij het onderwerp wil bespreken.
%% Vergeet ook niet te bedanken wie je geholpen/gesteund/... heeft

De afgelopen jaren heeft AI een enorme ontwikkeling doorgemaakt. Er zijn een groot aantal AI's ontwikkeld die goed zijn in het uitvoeren van specifieke taken, enkele voorbeelden zijn beeldgeneratie, taalvertaling en spraakherkenning. Deze AI's worden ook wel narrow AI's genoemd, omdat ze beperkt zijn in het combineren van taken.

De ontwikkeling van artificiele intelligentie en machine learning heeft de afgelopen jaren grote sprongen vooruit gemaakt. Zo hebben slimme assistenten als Siri, Alexa en Google Assistant ons leven drastisch veranderd door ons te helpen met dagelijkse taken, het beantwoorden van vragen en het aansturen van slimme apparaten. Maar wat als we nog verder zouden gaan, en in staat zouden zijn om software te ontwikkelen die het gevoel van een echte conversatie kan nabootsen, op vlak van zowel spraak, beeld als kennis?

In deze bachelorproef neem ik u mee door de verschillende AI-technologieën die gebruikt kunnen worden om dit doel te bereiken. Ik beschrijf hoe deze technologieën werken, hoe ze kunnen worden gecombineerd en hoe ze kunnen worden toegepast om de gebruiker het gevoel te geven dat hij of zij een echte conversatie heeft met een echt persoon.

Mijn motivatie om deze bachelorproef te schrijven is vooral gebaseerd op de recente evolutie in het domein van AI, aangezien er steeds meer AI's worden ontwikkeld die zaken genereren die vervolgens moeilijk van menselijke creaties te onderscheiden zijn. Door te onderzoeken welke combinaties van ANI's nodig zijn om het bovenstaande te simuleren, hoop ik duidelijkheid te kunnen scheppen over dit thema aan de hand van een concrete case.

Ik hoop dat deze bachelorproef bijdraagt aan het inzicht in de mogelijkheden en beperkingen van narrow AI-technologieën en hoe deze kunnen worden toegepast om een meer menselijke conversatie-ervaring te bieden.

Ik wil graag mijn dank uitspreken aan mijn begeleider en alle andere mensen die mij hebben geholpen bij het uitvoeren van dit onderzoek. Dit omvat niet alleen iedereen die mij hebben voorzien van waardevolle informatie en inzichten, maar ook iedereen die mij heeft gesteund tijdens de soms uitdagende momenten van dit proces.


%%=============================================================================
%% Samenvatting
%%=============================================================================

% TODO: De "abstract" of samenvatting is een kernachtige (~ 1 blz. voor een
% thesis) synthese van het document.
%
% Deze aspecten moeten zeker aan bod komen:
% - Context: waarom is dit werk belangrijk?
% - Nood: waarom moest dit onderzocht worden?
% - Taak: wat heb je precies gedaan?
% - Object: wat staat in dit document geschreven?
% - Resultaat: wat was het resultaat?
% - Conclusie: wat is/zijn de belangrijkste conclusie(s)?
% - Perspectief: blijven er nog vragen open die in de toekomst nog kunnen
%    onderzocht worden? Wat is een mogelijk vervolg voor jouw onderzoek?
%
% LET OP! Een samenvatting is GEEN voorwoord!

%%---------- Nederlandse samenvatting -----------------------------------------
%
% TODO: Als je je bachelorproef in het Engels schrijft, moet je eerst een
% Nederlandse samenvatting invoegen. Haal daarvoor onderstaande code uit
% commentaar.
% Wie zijn bachelorproef in het Nederlands schrijft, kan dit negeren, de inhoud
% wordt niet in het document ingevoegd.

\IfLanguageName{english}{%
\selectlanguage{dutch}
\chapter*{Samenvatting}



\selectlanguage{english}
}{}

%%---------- Samenvatting -----------------------------------------------------
% De samenvatting in de hoofdtaal van het document

\chapter*{\IfLanguageName{dutch}{Samenvatting}{Abstract}}

\lipsum[1-4]


%---------- Inhoudstafel -------------------------------------------------------
\pagestyle{empty} % Geen hoofding
\tableofcontents  % Voeg de inhoudstafel toe
\cleardoublepage  % Zorg dat volgende hoofstuk op een oneven pagina begint
\pagestyle{fancy} % Zet hoofding opnieuw aan

%---------- Lijst figuren, afkortingen, ... ------------------------------------

% Indien gewenst kan je hier een lijst van figuren/tabellen opgeven. Geef in
% dat geval je figuren/tabellen altijd een korte beschrijving:
%
%  \caption[korte beschrijving]{uitgebreide beschrijving}
%
% De korte beschrijving wordt gebruikt voor deze lijst, de uitgebreide staat bij
% de figuur of tabel zelf.

\listoffigures
\listoftables
\lstlistoflistings

% Als je een lijst van afkortingen of termen wil toevoegen, dan hoort die
% hier thuis. Gebruik bijvoorbeeld de ``glossaries'' package.
% https://www.overleaf.com/learn/latex/Glossaries

%---------- Kern ---------------------------------------------------------------

% De eerste hoofdstukken van een bachelorproef zijn meestal een inleiding op
% het onderwerp, literatuurstudie en verantwoording methodologie.
% Aarzel niet om een meer beschrijvende titel aan deze hoofstukken te geven of
% om bijvoorbeeld de inleiding en/of stand van zaken over meerdere hoofdstukken
% te verspreiden!

%%=============================================================================
%% Inleiding
%%=============================================================================

\chapter{\IfLanguageName{dutch}{Inleiding}{Introduction}}
\label{ch:inleiding}

Met de recente ontwikkelingen in AI (GPT) en machine learning wordt het steeds meer mogelijk om software te ontwikkelen die het gevoel van een echte conversatie kan nabootsen, op vlak van zowel spraak, beeld als kennis. Dit opent de deur naar nieuwe en meer geavanceerde toepassingen, zoals het ontwikkelen van een software die een gesprek kan hebben met een persoon via een videocall en die nauwelijks van een echt persoon te onderscheiden is.

In deze bachelorproef wordt onderzocht welke combinaties van narrow AI's nodig zijn om dit doel te bereiken. Er wordt beschreven hoe verschillende AI-technologieën werken, hoe ze kunnen worden gecombineerd en hoe ze kunnen worden toegepast om de gebruiker het gevoel te geven dat hij of zij een echte conversatie heeft met een echt persoon. Dit onderzoek is niet alleen relevant voor developers die deze software zouden willen bouwen, maar ook voor gebruikers van het internet, omdat het hen inzicht kan geven in wat er mogelijk is en een waarschuwing kan geven dat niet elke interactie die je online hebt sowieso met een persoon is, maar ook met een AI kan zijn.

Door deze ontwikkelingen wordt het steeds belangrijker om te begrijpen hoe narrow AI's kunnen worden gecombineerd om de ervaring van menselijke conversatie na te bootsen. Deze bachelorproef hoopt bij te dragen aan een beter inzicht in de mogelijkheden van deze technologieën en hoe ze kunnen worden toegepast om een meer menselijke conversatie-ervaring te bieden.

\section{\IfLanguageName{dutch}{Onderzoeksdoelstelling}{Research objective}}
\label{sec:onderzoeksdoelstelling}

De doelstelling van dit onderzoek is om een overzicht te geven van enkele narrow AI-technologieën die nodig zouden zijn om applicatie te ontwikkelen dat in staat is om een realistische conversatie-ervaring te bieden die niet van menselijke interactie te onderscheiden is. Hierbij zal er onderzocht worden welke combinaties van AI's hiervoor geschikt zouden kunnen zijn en hoe deze gecombineerd kunnen worden om dit doel te bereiken.

Daarnaast zal er een enquête worden gehouden om de verschillende menselijke aspecten van de case te evalueren op belangrijkheid. 
Hierbij zal het publiek gevraagd worden om bij verschillende aspecten van de conversatie-ervaring zoals bijvoorbeeld op vlak van beeld: de gegeneerde persoon, of de gelaatsuitdrukkingen de belangrijkheid te scoren. Deze gegevens zullen worden geanalyseerd om te bepalen welke aspecten het belangrijkste zijn voor het creëren van een realistische conversatie-ervaring met een combinatie van AI-systemen.

\section{\IfLanguageName{dutch}{Opzet van deze bachelorproef}{Structure of this bachelor thesis}}
\label{sec:opzet-bachelorproef}

% Het is gebruikelijk aan het einde van de inleiding een overzicht te
% geven van de opbouw van de rest van de tekst. Deze sectie bevat al een aanzet
% die je kan aanvullen/aanpassen in functie van je eigen tekst.

De rest van deze bachelorproef is als volgt opgebouwd:

In Hoofdstuk~\ref{ch:literatuurstudie} wordt een overzicht gegeven van de stand van zaken binnen het onderzoeksdomein, op basis van een literatuurstudie.

In Hoofdstuk~\ref{ch:methodologie} wordt de methodologie toegelicht en worden de gebruikte onderzoekstechnieken besproken om een antwoord te kunnen formuleren op de onderzoeksvragen.

% TODO: Vul hier aan voor je eigen hoofstukken, één of twee zinnen per hoofdstuk

In Hoofdstuk~\ref{ch:conclusie}, tenslotte, wordt de conclusie gegeven en een antwoord geformuleerd op de onderzoeksvragen. Daarbij wordt ook een aanzet gegeven voor toekomstig onderzoek binnen dit domein.
\chapter{\IfLanguageName{dutch}{Stand van zaken}{State of the art}}
\label{ch:stand-van-zaken}

% Tip: Begin elk hoofdstuk met een paragraaf inleiding die beschrijft hoe
% dit hoofdstuk past binnen het geheel van de bachelorproef. Geef in het
% bijzonder aan wat de link is met het vorige en volgende hoofdstuk.

% Pas na deze inleidende paragraaf komt de eerste sectiehoofding.

Voor we het onderzoek kunnen voeren naar hoe we kunnen evolueren van een Artificial Narrow Intelligence naar een Artificial General Intelligence is het eerst belangrijk om grondig uit te leggen wat beide termen juist inhouden en welke onderzoeken er al zijn uitgevoerd in het domein dat we onderzoeken.

Eerst zal men de geschiedenis van AI even schetsen, en benadrukken waarom de evolutie van ANI naar AGI misschien niet zo ver in de toekomst ligt, zoals veel experts denken.

\section{Waar staat AI vandaag, en hoe snel gaat de evolutie?}
De term 'AI' is een term die we in het huidige tijdperk niet meer kunnen wegdenken. Het is alom bekend en wordt in veel (grote) bedrijven gebruikt voor allerhande toepassingen. Maar hoe zijn we hier geraakt? En is deze evolutie verlopen zoals we pakweg 20 jaar geleden zouden verwacht hebben?

\subsection{De populariteit van AI}
Volgens \cite{brynjolfsson2017artificial} dat zich vooral focust op AI in de bedrijfswereld is AI de dag van vandaag al enorm aanwezig in een enorm aantal bedrijven. Dit komt omdat er voortdurend 'general purpose' technologieën opduiken, dit zijn technologieën die de mogelijkheid hebben om een hele economie te kunnen beïnvloeden en het potentieel hebben om samenlevingen drastisch te veranderen door hun impact op al bestaande sociale en economische structuren. Neem nu bijvoorbeeld de verbrandingsmotor die er voor zorgde dat auto's, truck's,... gecommercialiseerd konden worden door bijvoorbeeld bedrijven zoals UPS of Uber. De meest belangrijke 'general purpose' technologie van deze tijd is machine learning: dit is het vermogen van een machine om te kunnen blijven evolueren en het voortdurend verbeteren van zijn prestaties zonder dat mensen exact moeten uitleggen hoe ze een bepaalde taak moeten volbrengen die hen gegeven wordt. 

Dit is dan ook deels de grootste factor waarom AI de dag van vandaag zo populair is, omdat er economisch, en dus binnen de bedrijfswereld ook enorm veel vooruitgang kan geboekt worden. Zeker de afgelopen jaren is er enorme vooruitgang geboekt om machines taken autonoom te laten uitvoeren.

\subsection{Wat kan AI momenteel al?}

AI, of Artificiële intelligentie wordt meestal gezien als de capaciteit van machines om menselijke taken uit te voeren, voornamelijk zaken die te maken hebben met congnitieve functies zoals luisteren, kijken, spreken,... 

Volgens enquetes van \cite{benbya2020artificial} wordt er gesuggereerd dat minder dan de helft van de organisaties zinvolle AI projecten hebben lopen, of het vooruitzicht hebben dat deze zullen plaatsvinden, dus hier is nog veel ruimte voor groei. 

Maar wat is het dan juist? Waar is AI nuttig voor, en waarvoor wordt het gebruikt? Dit zal hier verder besproken worden.

\subsubsection{Business}
Ook al blijven de meeste AI-projecten die opgezet worden in de bedrijfswereld experimenteel, of een proof of concept, is het niet zo dat er geen enkel bedrijf is waar AI gebruikt wordt. De grootste reden dat bedrijven hun AI-project niet uitrollen heeft te maken met de weinig opbrengsten die hiermee behaald konden worden.

Toch wordt AI effectief gebruikt en kan men uit een enquete, afgenomen door Deloitte, waar bedrijfsleiders werden gevraagd waarvoor ze AI gebruiken, het onderstaande afleiden (percentages betekenen hoe vaak de antwoorden zijn voorgekomen bv. 'om keuzes maken te vergemakkelijken' kwam voor in 1 op 4 van de enquetes, er konden tevens ook meerdere antwoorden worden geselecteerd)

\begin{itemize}
    \item 28\% om processen te vergemakkelijken
    \item 25\% om bestaande producten of services te verbeteren
    \item 23\% om nieuwe producten of services te creëren
    \item 21\% om keuzes maken te vergemakkelijken
    \item 20\% om kosten te verlagen
\end{itemize}

Een interessant gegeven is dat AI vaak genoemd wordt om het aantal werknemers te verminderen in een bedrijf, werd dit maar in 11\% van de enquetes genoemd. 

Echter in deze tendens wordt er nu een verschuiving gemerkt. Waar werkgevers initieel enkel focuste op het gebruiken van AI om bepaalde specifieke workflows, processen en repetitief werk te automatiseren, wordt er nu meer gekeken om AI in te zetten voor niet systematische congitieve taken, die zelfs keuzes kunnen maken en problemen kunnen oplossen. Zelfs creativiteit is iets waarvoor AI momenteel kan gebruikt worden. Dit is iets wat pakweg 5 jaar geleden buiten de scope van AI viel.

Hoe en waarom deze tendens gebroken werd zal later in deze literatuurstudie aan bod komen. 

Eerst is het belangrijk om een goed beeld te vormen van wat er momenteel (technologisch) mogelijk is met AI.

\subsubsection{Technologie}
Het is belangrijk om een beeld te scheppen van de evolutie van de technologie achter AI tot dusver.

Eerst is het belangrijk om te benadrukken dat de term 'AI' redelijk breed is.

Uit de studie van \cite{benbya2020artificial} kan men AI types bekijken van uit 3 inzichten:
\begin{itemize}
    \item Gebasseerd op functie
    \item Gebasseerd op technologie
    \item Gebasseerd op intelligentie
\end{itemize}

We kunnen deze 3 brede takken nog iets beter toelichten per tak.

Gebasseerd op functie: 

    Hier wordt het onderscheid gemaakt tussen vier soorten artificiële intelligentie. Namelijk conversationale, biometrische, algoritmische en robotische AI. Hier gaat het dus louter om de functie waarvoor de AI gebruikt wordt. 
    
    Zo is conversationele AI sinds de opkomst van OpenAI en ChatGPT zeer populair, dit is zoals het woord het wel al doet vermoeden, een ai die in staat is om menselijke taal te herkennen, en te begrijpen. Dit doormiddel van tekst -en stemherkenning. Conversationele AI heeft bijgevolg het meeste kans om meer complexe taken te kunnen uitvoeren, omdat het duidelijker kan gemaakt worden wat de opdracht is, als de AI het taalmodel begrijpt van de opdrachtgever, ook opnieuw is hier het beste voorbeeld de GPT4 technologie. Hier over later meer. 
    
    Biometrische AI gaat dan weer om met het fysiologische aspect van de mens en heeft als functie bijvoorbeeld vingerafdrukken herkennen, iris scanner,... maar ook bijvoorbeeld het herkennen van gedragskenmerken zoals een handtekening, stem,... 
    
    Algoritmische AI heeft voornamelijk te maken met machine learning (ML) algoritmen. Dit zijn een aantal instructies die een computer kan uitvoeren. Dit kan bijvoorbeeld zijn om een bepaalde specifieke taak aan te leren aan de AI.

    Robotische AI slaat op fysieke robots, deze worden al geringe tijd gebruikt in bijvoorbeeld fabrieken. Robots met AI kunnen hun omgeving waarnemen, begrijpen, leren en hier acties op ondernemen. Dit draagt bij aan het feit dat robots een groot aantal taken kunnen uitvoeren, dit kan gaan over mensen assistentie bieden bij een aantal taken, of het identificeren van objecten, en op basis van deze objecten bepaalde taken uitvoeren.

Gebasseerd op technologie:

    Hier is het onderscheid gebasseerd op de technologie die gebruikt wordt in het AI systeem. Dit gaat vooral over machine learning, deep learning, neural networks, natural language processing, rule-based expert systemen, robotic process automation en robots.
    
    Machine learning kan men opzich nog eens opdelen in 3 takken, namelijk reinforcement learning, supervised learning en unsupervised learning.
    Kort uitgelegd: Reinforced learning slaat op een AI dat kan leren uit ervaring, zonder menselijke invloed. Supervised learning is wanneer een AI leert van een dataset waarmee het getrained wordt. Unsupervised learning is wanneer een AI patronen kan herkennen in data dat niet gelabeled is, en waarvan het resultaat nog niet gekend is.
    
    Deep learning is een klasse van machine learning waarbij de AI kan leren zonder supervisie van een persoon. Het kan zichzelf trainen via zowel gelabelde als ongelabelde data. Een welgekend voorbeeld hier van is spraak -en afbeeldingsherkenning.
    
    Neural networks zijn algoritmes die proberen onderliggende relaties uit een dataset te halen via een proces dat de werking van het menselijk brein probeert na te bootsen.
    
    Natural language processing kwam daarnet al aan bod en slaat op het feit dat een AI de mogelijkheid heeft om een taal te herkennen en te begrijpen. bv. ChatGPT
    
    Rule based expert systems zijn systemen waarbij er een aantal gepredefineerde regels worden opgesteld door een persoon. Deze worden dan gevolgd door deze AI, vandaar de naam Rule based.
    
    Robotic process automation systemen zijn systemen die gebouwd zijn om bepaalde digitale taken te automatiseren, dit kan bijvoorbeeld het inwisselen van een credit kaart zijn, waar er een aantal processen dienen uitgevoerd worden, deze gebeuren dan via de AI.
    
    Robots kwamen daarnet ook al aan bod, en slaat op het automatisch bedienen van machines. Bijvoorbeeld in een fabriek.
    
Gebasseerd op intelligentie:

    Dit slaat op de intelligentie van de AI. En dit kan men opdelen in 3 delen, hiervan zijn 2 delen ook tevens de belangrijkste voor deze studie.
    \begin{itemize}
        \item Artificial Narrow Intelligence
        \item Artificial General Intelligence
        \item Artificial Super Intelligence
    \end{itemize}

    Artificial Narrow Intelligence (ANI) en Artificial General Intelligence (AGI) worden verder in deze literatuurstudie besproken, dus hier zal men momenteel niet te diep op in gaan. 
    
    Artificial Narrow Intelligence slaat op een AI dat vaak maar de mogelijkheid heeft om één bepaald probleem op te lossen. En niet in staat is om vaardigheden die het kent om dit bepaald probleem op te lossen, over te dragen naar een ander domein. 
    
    Artificial General Intelligence slaat op een AI die streeft naar het leren van zaken op menselijk niveau. Het kan vaardigheden van het ene domein naar het andere domein overdragen en kan ook linken leggen tussen bepaalde oplossingen voor gegeven problemen. Het kan leren zoals een mens dit kan.
    
    Artificial Super Intelligence is een theoretische vorm van een AI, waar het mogelijk zou zijn om de mensenlijke aspecten te overstijgen. De theorie zegt dat wanneer men er in slaagt om een AGI te ontwikkelen dat deze super intelligence zonder problemen slimmer, beter en meer capaciteiten heeft dan de mens. 


\subsection{Recente veranderingen in het domein}

\subsubsection{ChatGPT}

\subsubsection{Komt dit in de buurt van AGI?}

\section{Artificial Narrow Intelligence}

\subsection{Technische aspecten}

\subsection{Opzetten van een ANI}

\section{Artificial General Intelligence}

\subsection{Technische aspecten}

\subsection{Menselijke aspecten}




%%=============================================================================
%% Methodologie
%%=============================================================================

\chapter{\IfLanguageName{dutch}{Methodologie}{Methodology}}
\label{ch:methodologie}

%% TODO: Hoe ben je te werk gegaan? Verdeel je onderzoek in grote fasen, en
%% licht in elke fase toe welke stappen je gevolgd hebt. Verantwoord waarom je
%% op deze manier te werk gegaan bent. Je moet kunnen aantonen dat je de best
%% mogelijke manier toegepast hebt om een antwoord te vinden op de
%% onderzoeksvraag.

Om te kunnen onderzoeken hoe men een impressie zou kunnen wekken dat men met een echt persoon contact staat werd er gefocust op 2 belangrijke aspecten:

\begin{itemize}
    \item Welke narrow ai's dient men te combineren om bepaalde menselijke aspecten na te bootsten
    \item Belangrijkheid van deze aspecten bevragen
\end{itemize}

Eerst werd onderzocht worden welke narrow ai's een mogelijke combinatie zouden kunnen zijn voor verschillende aspecten op vlak van beeld, spraak en kennis. Daarna werd aan de hand van een enquete getoetst welke aspecten het belangrijkste zijn voor de gebruiker, om deze ervaring te verkrijgen.

\newpage

\section{Menselijke aspecten}

Uiteraard is het schetsen van alle eigenschappen van een mens onbegonnen werk, dit is bij elke persoon anders omdat iedereen andere karaktereigenschappen heeft. Ook zijn zaken zoals bijvoorbeeld een geweten of moraal één van de moeilijkste zaken om in een technologie te gieten. 

In dit onderzoek werd er op zoek gegaan naar de meest voor de hand liggende eigenschappen van een mens, niet bepaald het emotionele gegeven, dit kan misschien wel een leuke uitbreiding zijn op deze bachelorproef voor een bachelorproef in het psychologie veld.

Om dit onderzoek te schetsen ging men er van uit dat men een ai zou willen opzetten die via bijvoorbeeld een videocall met video en spraak contact zou hebben met een persoon, en dat de tegenpartij niet kan onderscheiden of er contact gemaakt wordt met een technologie of een echte persoon.

Bovenstaande case werd opgesplitst in de drie meest voor de hand liggende delen:

\begin{itemize}
    \item Beeld
    \item Spraak
    \item Kennis
\end{itemize}

In de volgende subsecties werd dit meer in detail onderzocht

\newpage

\section{Beeld}

Op vlak van beeld werd er van uit gegaan dat onderstaande zaken het belangrijkste waren:

\begin{itemize}
    \item Gezichtsherkenning
    \item Emotieherkenning
    \item Deepfakes
\end{itemize}

Gezichtsherkenning: Een AI die in staat is om gezichten te herkennen en te onderscheiden. Dit kan helpen om gepersonaliseerde en aangepaste conversaties te bieden, bijvoorbeeld door de herkenning van emoties op het gezicht van de gesprekspartner.

Emotieherkenning: Het vermogen om emoties te herkennen via het gelaat en te interpreteren is essentieel voor effectieve communicatie tussen mensen. Een AI die menselijke emoties kan herkennen en daarop kan reageren, kan helpen om een meer natuurlijke conversatie te creëren.

Deepfakes: Door het gebruik van deepfakes, kan de AI gezichtsuitdrukkingen en andere non-verbale signalen nabootsen, waardoor de conversatie nog overtuigender wordt.

\newpage

\subsection{Gezichtsherkenning}

Gezichtsherkenning is een technologie die het mogelijk maakt om individuele gezichten te identificeren en te verifiëren op basis van specifieke kenmerken, zoals de afstand tussen de ogen, de neus en de mond, de vorm van het gezicht en de verhoudingen tussen deze kenmerken.

Er zijn verschillende methoden voor gezichtsherkenning, waaronder 2D-gezichtsherkenning, 3D-gezichtsherkenning en gezichtsherkenningsalgoritmen gebaseerd op artificiele intelligentie (AI). 2D-gezichtsherkenning maakt gebruik van platte afbeeldingen van het gezicht om het te herkennen, terwijl 3D-gezichtsherkenning gebruik maakt van driedimensionale modellen van het gezicht om meer gedetailleerde en nauwkeurige identificatie te bieden.

Zoals hierboven besproken is dit interessant voor deze case om via deze manier gepersonaliseerde en aangepaste conversaties te bieden. Wanneer men focust op gezichtsherkenning gebasseerd op artificiele intelligentie dan komt men op  artificiele intelligentie (AI) die gebruik maakt van deep learning-technieken, waaronder convolutionele neurale netwerken (CNN's) en recurrente neurale netwerken (RNN's), om gezichten te herkennen en te identificeren.

\subsubsection{Onderzoek naar combinatie van Artificial Narrow Intelligences}

Wanneer men verschillende artificial narrow intelligences zou gaan willen combineren om gezichtsherkenning goed te laten verlopen denkt men in eerste instantie aan:

\begin{itemize}
    \item Computer vision
    \item Patroonherkenning
    \item Machine learning
\end{itemize}

In eerste instantie is het belangrijk dat dit onderdeel van de software in staat is om digitale beelden van gezichten te analyseren en verwerken. Via deze manier kan men belangrijke kenmerken van een gezicht, zoals bijvoorbeeld de ogen, neus en mond te identificeren. Dit is een onderdeel van computer vision.

Vervolgens zou men via patroonherkenning moeten kunnen herkennen welke combinaties van kenmerken bij een bepaald gezicht horen. Via deze gekende patronen kan men de identiteit van een persoon bepalen.

Bij zowel computer vision als patroonherkenning is machine learning een belangrijke tool om het gezichtsherkenningssysteem te verbeteren en te optimaliseren.

Zo kan machine learning bij computer vision worden gebruikt om het systeem te trainen om automatisch bepaalde kenmerken van een gezicht te herkennen, zoals de positie van de ogen, neus en mond. Hier kunnen convolutionele neurale netwerken voor worden gebruikt, die zeer handig zijn voor beeldverwerking.

Ook voor patroonherkenning kan machine learning worden gebruikt om het systeem te trainen om automatisch patronen in gezichtskenmerken te herkennen die uniek zijn voor een bepaald persoon.

\subsection{Emotieherkenning}

Emotieherkenning is het proces waarbij software of technologie wordt gebruikt om menselijke emoties te detecteren en te interpreteren op basis van gezichtsuitdrukkingen, stemintonatie en andere fysieke signalen.

Het maakt gebruik van verschillende methoden, waaronder beeld- en spraakverwerking, machine learning en artificiele intelligentie om emoties te identificeren en te classificeren. In het geval van emotieherkenning wordt gebruik gemaakt van computeralgoritmen om gezichtsuitdrukkingen te analyseren en te interpreteren, zoals glimlachen, fronsen en het bewegen van de wenkbrauwen.

Binnen emotieherkenning wordt artificiele intelligentie (AI) gebruikt om complexe patronen in de gegevens te ontdekken en te interpreteren. AI-algoritmen kunnen worden getraind om gezichtsuitdrukkingen te herkennen die specifiek zijn voor bepaalde emoties.

De meest gebruikte methode binnen AI voor emotieherkenning is machine learning. Hierbij worden de algoritmen getraind op een grote hoeveelheid gegevens om patronen te ontdekken en voorspellingen te doen over nieuwe gegevens. Het trainen van de algoritmen gebeurt meestal met behulp van datasets die handmatig zijn gelabeld door menselijke experts. Deze labels worden gebruikt om het algoritme te leren welke kenmerken overeenkomen met welke emoties.

Naast machine learning zijn er ook andere AI-technieken die worden gebruikt in emotieherkenning, zoals deep learning en neurale netwerken. Deze methoden maken gebruik van complexe wiskundige modellen die zijn ontworpen om grote hoeveelheden gegevens te verwerken en complexe patronen te identificeren.

\subsubsection{Onderzoek naar combinatie van Artificial Narrow Intelligences}

Wanneer men verschillende artificial narrow intelligences zou gaan willen combineren om emotieherkenning goed te laten verlopen denkt men in eerste instantie aan:

\begin{itemize}
    \item Computer vision
    \item Patroonherkenning
    \item Machine learning
\end{itemize}

De benodigde articial narrow intelligences die men dient te combineren zijn in eerste instantie dezelfde als bij gezichtsherkenning, omdat ze beide steunen op het visuele aspect.

Zo zal computer vision dienen om digitale beelden van gezichten te analyseren en verwerken om zo kenmerken in verband met gezichtsuitdrukkingen, zoals oog-, mond- en wenkbrauwpositie te identificeren.

Patroonherkenning zal dan op zijn beurt moeten kunnen herkennen welke patronen in gezichtsuitdrukkingen bij bepaalde emoties horen en deze patronen kunnen identificeren om op deze manier de emotie van een persoon te achterhalen.

Machine learning zorgt er op zijn beurt dan weer voor dat men de AI kan trainen via voorbeelden van emoties in gezichtsuitdrukkingen, om zo emoties nauwkeuriger te herkennen en te classificeren.

Dankzij deze combinatie kan een emotieherkenningsysteem worden gemaakt dat in staat is om emoties nauwkeurig te identificeren en te classificeren op basis van gezichtsuitdrukkingen.

\subsection{Deepfake}

Een deepfake is een type vervalste video, afbeelding of audio waarbij artificiele intelligentie (AI) wordt gebruikt om de inhoud te manipuleren en te creëren. De term 'deepfake' komt van de combinatie van 'deep learning' en 'fake'.

Deepfakes worden gemaakt met behulp van machine learning-algoritmen, zoals generatieve neurale netwerken, om een ​​realistische, vervalste weergave te creëren van iemand die niet echt bestaat.

Deepfakes worden vaak gecombineerd met andere machine learning-technieken zoals gezichts- en stemherkenning.

\subsubsection{Onderzoek naar combinatie van Artificial Narrow Intelligences}

Wanneer men verschillende artificial narrow intelligences zou gaan willen combineren om een deepfake te creëren denkt men in eerste instantie aan het trainen van een deepfake via een generatief antagonisten netwerk.

Voor het tonen van een beeld dat gegenereerd is en op een echt persoon moet lijken denkt men voornamelijk aan deepfakes. Wanneer men bij een deepfake focust op enkel het beeld dan kan een deepfake getrained worden via een Generatief Antagonisten Netwerk.

Deze AI kan worden gebruikt om nieuwe en realistische beelden te genereren. Via deze weg kan men realistische deepfakes creëren waar het voor de gebruiker van de software mogelijks niet merkbaar is dat deze door een AI gegenereerd is.

\section{Spraak}

Op vlak van spraak werd er van uit gegaan dat onderstaande zaken het belangrijkste waren:

\begin{itemize}
    \item Spraakherkenning
    \item Spraakverwerking
    \item Taalherkenning
    \item Taalverwerking
    \item Synthetische stemtechnologie
\end{itemize}

Spraakherkenning: Een AI die in staat is om spraak te herkennen kan helpen om een meer natuurlijke conversatie te creëren. Door de AI in staat te stellen om spraak te herkennen, kan het beter reageren op de gesproken vragen en opmerkingen van de tegenpartij.

Spraakverwerking: Dit gaat om een AI die in staat is om grammatica en woordbetekenissen te begrijpen. Ook hoort het begrijpen van context en intentie hier bij. Dit spreekt voor zich dat dit van groot belang is voor het creëren van een communicatie die natuurlijk aanvoelt.

Taalherkenning: De mogelijkheid om taal te herkennen is essentieel voor het ontwikkelen van een AI die menselijke communicatie kan nabootsen. Door de AI in staat te stellen taal te herkennen, kan het beter begrijpen wat de tegenpartij zegt en vervolgens de juiste reacties geven.

Taalverwerking: Taalverwerking is de volgende stap na taalherkenning. Het is het vermogen van de AI om de taal van de gesprekspartner te interpreteren en er op te reageren. Een effectieve taalverwerking is van enorm belang om een natuurlijke en intuïtieve conversatie te creëren.

Synthetische stemtechnologie: Door synthetische stemtechnologie te gebruiken, kan de AI menselijke spraak nabootsen, waardoor de conversatie overtuigender en realistischer wordt. Dit slaat voornamelijk op menselijk klinken. Het gebruik van synthetische stemtechnologie is daarom ook zeer belangrijk.

\newpage

\subsection{Spraakherkenning}

Spraakherkenning, is een technologie waarmee computers menselijke spraak kunnen begrijpen en interpreteren. Het proces van spraakherkenning omvat het opnemen van spraakgeluiden, deze omzetten in elektrische signalen en vervolgens de omzetting van deze elektrische signalen in digitale gegevens die kunnen worden begrepen en geanalyseerd door de computer.

Er zijn twee soorten spraakherkenning: onafhankelijke spraakherkenning en spraakherkenning met grammatica. Onafhankelijke spraakherkenning is de eenvoudigste vorm, waarbij de computer naar specifieke woorden luistert en deze herkent. Spraakherkenning met grammatica is complexer en wordt gebruikt om complete zinnen te herkennen. Het vereist een taalmodel dat bekend is met de grammatica en de vocabulaire van de gesproken taal.

\subsubsection{Onderzoek naar combinatie van Artificial Narrow Intelligences}

Wanneer men verschillende artificial narrow intelligences zou gaan willen combineren om spraakherkenning te bereiken denkt men in eerste instantie aan:

\begin{itemize}
    \item Digitale signaalverwerking
    \item Spraakherkenning
    \item Machine Learning
\end{itemize}

Om er voor te kunnen zorgen dat het systeem de audiosignalen die binnekomen door spraak van de gebruik kan vastleggen en verwerken kan men gebruik maken van digitale spraakverwerking. Dit kan ook helpen om achtergrondgeluiden uit de invoer te filteren en eventueel spraak te versterken. Een goede verwerking van de digitale signalen is cruciaal omdat andere AI's die gebruikt worden in de combinatie ook dit spraaksignaal verder dienen te analyseren.

Het spraakherkenningsysteem werkt verder met de spraaksignalen die verkregen zijn uit de digitale signaalverwerking. Deze AI is getrained om gesproken taal te herkennen en te transcriberen. Het wordt vaak gebruikt om de spraak in audio-opnames om te zetten in tekst, waardoor het makkelijker wordt om het te analyseren en verder te verwerken. Het systeem maakt gebruik van van verschillende algorithmes en modellen om het spraaksignaal te analyseren en te vergelijken met bekende spraakpatronen. Op deze manier probeert de AI woorden te identificeren en deze om te zetten in tekst. Dankzij digitale signaalverwerking kan het spraakherkenningssysteem beter in staat zijn om het spraaksignaal nauwkeurig te begrijpen.

Machine learning wordt dan weer gebruikt om bijvoorbeeld het spraakherkenningsmodel te trainen. Het kan helpen om de nauwkeurigheid te verbeteren door modellen te voorzien met meer trainingsdata.

Door deze artificial narrow intelligences te combineren, kan men een spraakherkenningssysteem creëren dat in staat is om menselijke spraak te begrijpen en te transcriberen. Het is belangrijk om op te merken dat de nauwkeurigheid van spraakherkenningssystemen afhankelijk is van de kwaliteit van de audio-opnames en de trainingsgegevens die worden gebruikt om de modellen te trainen.

\subsection{Spraakverwerking}

Spraakverwerking is een technologie die de computer in staat stelt om natuurlijke spraak te begrijpen, te genereren en te transformeren. Het omvat verschillende processen, zoals spraakherkenning, spraaksynthese, spraakvertaling en spraakanalyse.

Spraakverwerking begint met spraakopname, die men in dit geval bekomt via spraakherkenning, waarbij de stem van een spreker wordt omgezet in digitale signalen door middel van microfoons en signaalverwerkingstechnieken. Deze digitale signalen worden vervolgens geanalyseerd en verwerkt door de computer om de spraakinformatie te extraheren en te begrijpen.

Spraakherkenning is een van de belangrijkste toepassingen van spraakverwerking. Het omvat het omzetten van gesproken woorden en zinnen in tekstuele vorm, wat nuttig is voor spraakgestuurde systemen, zoals virtuele assistenten, spraak-naar-tekst transcriberen, etc. Zonder spraakherkenning kan men niet verder met de verwerking.

Spraakverwerking maakt gebruik van verschillende geavanceerde technologieën, zoals machine learning, deep learning en neurale netwerken. Deze technologieën hebben spraakverwerking nauwkeuriger en efficiënter gemaakt, waardoor deze technologie kan worden toegepast in diverse domeinen en toepassingen.

\subsubsection{Onderzoek naar combinatie van Artificial Narrow Intelligences}

Wanneer men verschillende artificial narrow intelligences zou gaan willen combineren om spraakverwerking te bereiken denkt men in eerste instantie aan:

\begin{itemize}
    \item Taalherkenning
    \item Taalverwerking
\end{itemize}

Door deze ANI's te combineren, kunnen systemen voor spraakverwerking spraak omzetten in tekst, de betekenis en context van de spraak begrijpen, en vervolgens deze kennis gebruiken om tekst om te zetten in natuurlijke spraak.

\subsubsection{Taalherkenning}

Taalherkenning is een technologie die computers in staat stelt om natuurlijke taal te begrijpen en te verwerken.

Taalherkenning begint met het identificeren en begrijpen van de structuur van de zin, inclusief de grammaticale en semantische betekenis van de woorden. Dit omvat het identificeren van de onderwerpen, werkwoorden, objecten en andere belangrijke elementen van de zin. Dit proces is afhankelijk van de beschikbaarheid van taalkundige regels, woordenboeken en taalmodellen.

Taalherkenning maakt gebruik van verschillende technologieën, zoals machine learning en deep learning. Deze technologieën maken gebruik van complexe algoritmes en modellen die taal kunnen begrijpen en verwerken, waardoor computers steeds beter worden in het begrijpen van natuurlijke taal.

\subsubsection{Taalverwerking}

Tijdens dit proces proberen computers natuurlijke taal te begrijpen en verwerken. Dit proces omvat verschillende taken, zoals onder andere taalanalyse, natuurlijke taalverwerking, taalgeneratie en taalvertaling.

Taalanalyse is het proces waarbij computers de inhoud van teksten analyseren om inzicht te krijgen in verschillende aspecten, zoals onderwerp, sentiment, en intentie

Taalverwerking maakt gebruik van verschillende technologieën, zoals machine learning, deep learning en natuurlijke taalverwerking (NLP). Deze technologieën maken gebruik van complexe algoritmes en modellen die taal kunnen begrijpen en verwerken, waardoor computers steeds beter worden in het begrijpen en produceren van natuurlijke taal.
 
Taalgeneratie is het proces waarbij computers tekstuele informatie omzetten in natuurlijke taal.

Taalvertaling is een andere belangrijke toepassing van taalverwerking en maakt het mogelijk om tekstuele informatie van de ene taal naar de andere te vertalen.

\subsubsection{Onderzoek naar combinatie van Artificial Narrow Intelligences}

Wanneer men verschillende artificial narrow intelligences zou gaan willen combineren om taalverwerking te bereiken denkt men in eerste instantie aan:

\begin{itemize}
    \item Taalanalyse
    \item Natuurlijke taalverwerking
    \item Taalgeneratie
    \item Taalvertaling
\end{itemize}

In eerste instantie dient men de herkende tekst te analyseren en segmenteren in woorden, zinnen, grammaticale structuur, semantische betekenis, etc. Het kan ook betrekking hebben op het identificeren van specifieke taalkenmerken, zoals woordsoorten, leestekens, hoofdletters,...

Eenmaal dit proces voorbij is kan men algoritmes en regels gebruiken om de betekenis en context van menselijke taal te begrijpen. Het omvat het verwerken van zinnen, het identificeren van relaties tussen woorden en zinnen, het herkennen van entiteiten en het extraheren van belangrijke informatie. Dit is natuurlijke taalverwerking.

Van zodra de computer de betekenis begrijpt kan het starten met taal te genereren. Dit kan gebeuren door middel van sjablonen, door regels gebaseerde methoden of door middel van machine learning-modellen.

Uiteindelijk kan het ook handig zijn om de gegenereerde tekst te vertalen naar andere talen, hiervoor is taalvertaling gepast. Het proces omvat meestal een combinatie van statistische en regelgebaseerde modellen die de grammatica, semantiek en syntaxis van beide talen begrijpen om nauwkeurige vertalingen te produceren.

Door het combineren van deze ANI's kan een systeem worden gecreëerd dat in staat is om menselijke taal te begrijpen, te analyseren, te genereren en te vertalen, en zo een breed scala aan taken uit te voeren die verband houden met taalverwerking.

\subsection{Synthetische stemtechnologie}

Synthetische stemtechnologie (ook wel text-to-speech of TTS genoemd) is een vorm van artificiele intelligentie (AI) die computers in staat stelt om natuurlijke spraak te produceren uit tekstuele input. Het proces begint met de invoer van tekstuele informatie, waarna het TTS-algoritme deze informatie analyseert en omzet in fonemen, die de kleinste geluidseenheden van de taal zijn. Vervolgens worden deze fonemen gecombineerd en uitgesproken als spraakgeluiden om een natuurlijke spraak te creëren.

Er zijn verschillende soorten synthetische stemtechnologieën, deze varieeren van eenvoudige systemen die standaardstemmen gebruiken tot meer geavanceerde systemen die gebruikmaken van deep learning-algoritmes om menselijke spraak te simuleren en een breder scala aan stemvariaties te bieden. Bovendien kunnen sommige systemen worden getraind op specifieke stemmen of accenten om een nog natuurlijkere spraak te produceren.

\subsubsection{Onderzoek naar combinatie van Artificial Narrow Intelligences}

Wanneer men verschillende artificial narrow intelligences zou gaan willen combineren om synthetische stemtechnologie te bereiken denkt men in eerste instantie aan:

\begin{itemize}
    \item Text-to-speech
    \item Spraakherkenning
    \item Emotionele spraakherkenning
\end{itemize}

Om er voor te zorgen dat het systeem kan spreken met de eindgebruiker, is het van belang dat de stem van de software zo goed mogelijk is. Om dit te kunnen bereiken kan men gebruik maken van verschillende artificial narrow intelligences die tekst kunnen omzetten naar spraak, en hier beter in blijven worden.

Eerst is het belangrijk om tekst te kunnen omzetten in gesproken woorden, hier is text-to-speech de ideale oplossing voor. Deze AI gebruikt neurale text-to-speech modellen om de spreekstijl van een natuurlijke stem te kunnen nabootsen. 

Spraakherkenning kan in dit geval worden gebruikt om de nauwkeurigheid van het text-to-speech te verbeteren, door feedback te geven over de gegenereerde text-to-speech. Op deze manier kan de text-to-speech zich blijven verbeteren.

Verder is emotionele spraakherkenning ook niet onbelangrijk om een menselijke conversatie na te bootsen. Deze AI is in staat om emoties in menselijke spraak te herkennen en kan gebruikt worden om synthetische spraakgeneratie meer expressief te maken. Via deze manier kan men emoties uitdrukken in de gegenereerde spraak.

Door deze artificial narrow intelligences te combineren, kunnen machines synthetische stemmen genereren die bijna niet van natuurlijke menselijke stemmen te onderscheiden zijn.

\section{Kennis}

Op vlak van kennis werd er van uit gegaan dat onderstaande zaken het belangrijkste waren:

\begin{itemize}
    \item Algemene kennis en skills verwerven en verbeteren
    \item Persoonlijkheidsmodellering
    \item Contextueel begrip
\end{itemize}

Kennis en vaardigheden: Het verwerven en verbeteren van kennis en vaardigheden is essentieel voor het ontwikkelen van een AI die menselijke communicatie kan nabootsen. Door de AI te programmeren met uitgebreide kennis en vaardigheden, kan het beter reageren op de vragen en opmerkingen van de gesprekspartner en kan het een meer overtuigende communicatie tot stand brengen.

Persoonlijkheidsmodellering: Een belangrijk aspect van effectieve communicatie is dat de gesprekspartner de persoonlijkheid van de ander begrijpt. Een AI die is geprogrammeerd om de persoonlijkheid van de gesprekspartner te modelleren, kan helpen om een meer gepersonaliseerde en aangepaste conversatie te bieden.

Contextueel begrip: Mensen gebruiken vaak impliciete contextuele aanwijzingen om de betekenis van gesproken taal te begrijpen. Een AI die in staat is om de context van een gesprek te begrijpen en daarop te reageren is dus curciaal voor deze case. Dit is tevens een onderdeel van spraakverwerking, maar blijft ook op vlak van kennis zeer belangrijk.

\newpage

\subsection{Algemene kennis en skills verwerven en verbeteren}

Een AI die zijn algemene kennis en vaardigheden kan verbeteren, wordt vaak aangeduid als een 'zelflerend' of 'zelfverbeterend' AI-systeem. Dit type AI maakt gebruik van machine learning-technieken, zoals deep learning en reinforcement learning, om zelfstandig te leren en te groeien in kennis en vaardigheden.

Om een echt gesprek met een mens te kunnen hebben, zou deze AI in staat moeten zijn om te communiceren met de gebruiker, zowel via spraak als tekst, en om verschillende taken uit te voeren. Het AI-systeem zou moeten kunnen begrijpen wat de gebruiker zegt en de intentie achter de woorden begrijpen, om zo de beste actie te bepalen die moet worden ondernomen. Het zou ook kennis moeten hebben van verschillende onderwerpen, zoals nieuws, weer, sport, etc, om op verzoek relevante informatie te kunnen geven.

Om zijn kennis en vaardigheden te verbeteren, zou het AI-systeem gebruik kunnen maken van verschillende technieken, zoals het bijhouden van logboeken van eerdere interacties met de gebruiker en het analyseren van deze gegevens om het systeem te verbeteren. Ook kan het systeem leren van nieuwe informatiebronnen, zoals het lezen van online artikelen en het bekijken van video's om zijn kennis te vergroten.

\subsubsection{Onderzoek naar combinatie van Artificial Narrow Intelligences}

Wanneer men verschillende artificial narrow intelligences zou gaan willen combineren om algemene kennis en skills te verwerven en te kunnen verbeteren denkt men in eerste instantie aan:

\begin{itemize}
    \item Machine Learning
    \item Natuurlijke taalverwerking
    \item Computer vision
    \item Expertsystemen
\end{itemize}

Om de algemene kennis en skills van een systeem te kunnen verwerven en blijven uit te breiden komen er heel wat artificial narrow intelligences bij kijken. Het is niet de bedoeling om een systeem te schetsen dat een artificial general intelligence is, wel is het zo dat men hier een impressie van wilt geven.

In eerste instantie is machine learning een belangrijk aspect in elk van deze processen. Het is zo dat machine learning wordt gebruikt om machines te trainen en om patronen in data te herkennen. Dit wordt gebruikt om AI's te leren hoe ze bepaalde taken moeten uitvoeren, en hoe ze met de opgedane kennis diezelfde prestaties kunnen verbeteren.

Vervolgens is natuurlijke taalverwerking ook hier van belang. Aangezien natuurlijke taalverwerking de machine in staat stelt om natuurlijke taal te kunnen verwerken en begrijpen. Dit is noodzakelijk om uit verschillende bronnen informatie te kunnen extraheren, die ook bijdraagt aan de algemene kennis van het systeem. Wel dient men op te letten welke bronnen men voedt, zo moet men bewaken dat de info correct en niet haatdragend of rasistisch kan zijn. Hier lijkt het de beste aanpak te zijn om erkende bronnen te voeden.

Computer vision wordt gebruikt om visuele informatie te kunnen begrijpen en verwerken. Denk bijvoorbeeld aan het herkennen van objecten. Dit kan handig zijn om het systeem niet meteen door de mand te doen vallen wanneer bijvoorbeeld de gebruiker iets vraagt over een item dat getoond wordt via de camera. 

Ook beslissingen kunnen nemen, en probleemoplossend denken zijn van belang voor deze case. Hier zijn expertsystemen een goede oplossing. Expertsystemen beheersen kennis uit een specifiek domein, en ze kunnen redeneringen en probleemoplossingen uitvoeren binnen dit domein. Reinforcement kan hier worden gebruikt om dit te trainen. Het zorgt er immers voor dat AI's kunnen leren via de trial and error methode. Zo wordt de AI beloond wanneer een taak op de juiste manier wordt uitgevoerd en gestraft wanneer een taak niet op de juiste manier wordt uitgevoerd. Dus in dit geval zal de AI beloont worden wanneer het een juiste beslissing of redenering heeft gemaakt, en gestraft worden wanneer dit niet zo is.

Door deze ANI's te combineren, kunnen machines worden getraind om kennis en vaardigheden te verwerven en te verbeteren.

\subsection{Persoonlijkheidsmodellering}

Persoonlijkheidsmodellering is het proces van het analyseren en voorspellen van menselijke persoonlijkheden op basis van verzamelde gegevens. Dit proces wordt vaak ondersteund door machine learning-algoritmen.

Om een persoonlijkheidsmodel te maken, worden gegevens verzameld over iemands gedrag, interesses, attitudes, waarden en andere persoonlijke kenmerken. Deze gegevens worden vervolgens geanalyseerd en georganiseerd in verschillende aspecten van persoonlijkheid, zoals extraversie, vriendelijkheid, openheid voor ervaringen, emotionele stabiliteit en consciëntieusheid.

Op basis van deze aspecten en de bijbehorende gegevens kunnen AI-algoritmen patronen en verbanden identificeren die kunnen worden gebruikt om persoonlijkheidsprofielen op te stellen en toekomstig gedrag te voorspellen. Deze profielen kunnen vervolgens worden gebruikt om geautomatiseerde persoonlijke assistenten en chatbots te creëren die zich kunnen aanpassen aan de persoonlijkheid van een gebruiker en betere, meer gepersonaliseerde interacties te kunnen bieden. Ideaal voor deze case dus.

\subsubsection{Onderzoek naar combinatie van Artificial Narrow Intelligences}

Wanneer men verschillende artificial narrow intelligences zou gaan willen combineren om persoonlijkheidsmodellering te bekomen denkt men in eerste instantie aan:

\begin{itemize}
    \item Natuurlijke taalverwerking
    \item Computer vision
    \item Recommendatie systemen
    \item Machine learning
\end{itemize}

Door deze ANI's te combineren, kunnen machines worden getraind om persoonlijkheidstypen te herkennen en te voorspellen op basis van verschillende gegevensbronnen en signalen.

In eerste instantie kan men een persoonlijkheid modelleren aan de hand van natuurlijke taalverwerking. Omdat natuurlijk taalverwerking in staat is om taal te begrijpen en verwerken. Dit is handig omdat persoonlijkheid vaak tot uiting komt in de manier hoe een persoon communiceert.

Computer vision is ook hier gepast omdat het kan gebruikt worden om non-verbale signalen te herkken, denk aan gezichtsuitdrukingen en lichaamstaal.

Om gepast te kunnen reageren op bepaalde herkende emoties of persoonlijkheden kan men gebruik maken van recommendatie systemen. Deze kunnen worden gebruikt om suggesties te doen op basis van het herkende gedrag van de gebruiker. Dit kan worden gebruikt om een bepaalde persoonlijkheid te suggereren op basis van voordien vertoonde voorkeuren en interesses.

Uiteindelijk is machine learning opnieuw een belangrijk onderdeel in deze sectie. Dit kan worden gebruikt om de AI's te leren persoonlijkheidskenmerken te herkennen aan de hand van verschillende databronnen, denk maar aan bijvoorbeeld taalgebruik, opinies, etc.

\subsection{Contextueel begrip}

Contextueel begrip verwijst naar het vermogen van een computerprogramma of AI-systeem om de context te begrijpen waarin taal of informatie wordt gebruikt en om de betekenis ervan te interpreteren op basis van deze context.

Contextueel begrip wordt meestal bereikt door gebruik te maken van natuurlijke taalverwerking en machine learning-technieken om woorden en zinnen te analyseren en te vergelijken met andere voorbeelden van vergelijkbare contexten. Door deze vergelijkingen kan een AI-systeem de betekenis van woorden en zinnen beter begrijpen en contextueel relevante antwoorden en acties produceren.

\subsubsection{Onderzoek naar combinatie van Artificial Narrow Intelligences}

Wanneer men verschillende artificial narrow intelligences zou gaan willen combineren om contextueel begrip te bekomen denkt men in eerste instantie aan:

\begin{itemize}
    \item Natuurlijke taalverwerking
    \item Knowledge graphs
    \item Recommendatie systemen
    \item Machine learning
\end{itemize}

Elk van deze zaken is reeds besproken, buiten knowledge graphs. Men zal toelichten waarom ze hier van toepassing zijn.

Natuurlijke taalverwerking blijft interessant omdat ze in staat zijn tekst te begrijpen. Via deze manier is het mogelijk om contextuele informatie te herkennen, zoals bijvoorbeeld de intentie van de gebruiker.

Knowledge graphs zijn nog niet aan bod gekomen, echter zijn ze niet onbelangrijk voor de totaliteit van deze case. Knowledge graphs zijn AI systemen die gebruik maken van machine learning om relaties tussen verschillende concepten te begrijpen. Dit is niet onbelangrijk om een fatsoenlijke conversatie te kunnen hebben met de gebruiker.

Ook recommendatie systemen komen hier te pas. Deze kunnen worden gebruikt om suggesties te doen die relavant zijn voor de huidige context.

Door deze ANI's te combineren, kunnen machines contextueel begrip ontwikkelen en gebruiken om betere beslissingen te nemen en meer zinvolle interacties te hebben met mensen.


% Voeg hier je eigen hoofdstukken toe die de ``corpus'' van je bachelorproef
% vormen. De structuur en titels hangen af van je eigen onderzoek. Je kan bv.
% elke fase in je onderzoek in een apart hoofdstuk bespreken.

%\input{...}
%\input{...}
%...

%%=============================================================================
%% Conclusie
%%=============================================================================

\chapter{Conclusie}
\label{ch:conclusie}

% TODO: Trek een duidelijke conclusie, in de vorm van een antwoord op de
% onderzoeksvra(a)g(en). Wat was jouw bijdrage aan het onderzoeksdomein en
% hoe biedt dit meerwaarde aan het vakgebied/doelgroep? 
% Reflecteer kritisch over het resultaat. In Engelse teksten wordt deze sectie
% ``Discussion'' genoemd. Had je deze uitkomst verwacht? Zijn er zaken die nog
% niet duidelijk zijn?
% Heeft het onderzoek geleid tot nieuwe vragen die uitnodigen tot verder 
%onderzoek?

\section{Onderzoek naar combinaties van artificial narrow intelligences om menselijke aspecten na te bootsen}

Deze bachelorproef heeft aangetoond dat het combineren van verschillende artificial narrow intelligences een potentieel succesvolle aanpak kan zijn om een conversatie-ervaring te creëren die de indruk geeft dat men met een persoon praat.

Bij beeld werd er gekozen voor gezichtsherkenning en emotieherkenning om een meer gepersonaliseerde ervaring te bieden aan de gebruiker. Deepfakes zijn dan de logische keuze om een beeld te genereren van een persoon die niet bestaat.

Voor spraak werd er voor verschillende AI-technologieën gekozen zoals spraakherkenning, spraakverwerking, taalherkenning en taalverwerking om de gebruiker een natuurlijke en vloeiende conversatie-ervaring te bieden. Synthetische stemtechnologie kan gebruikt worden om de interactie nog realistischer te maken.

Bij kennis werd gekozen voor het verwerven en verbeteren van algemene kennis en vaardigheden van de gebruiker. Ook persoonlijkheidsmodellering om de ervaring verder te personaliseren. Daarnaast werd contextueel begrip toegevoegd om de gebruiker beter te begrijpen en een meer gerichte respons te kunnen geven.

Een combinatie van deze verschillende AI-technologieën stelt ons in staat om een realistische en gepersonaliseerde conversatie-ervaring te schetsen, die vergelijkbaar is met een menselijke interactie. Het kan echter nog verder worden ontwikkeld en geoptimaliseerd om een nog betere gebruikerservaring te bieden.
Onderzoek naar combinaties van artificial narrow intelligences.

\section{Bevraging naar de belangrijkheid van deze aspecten}

(resultaat enquete ook bespreken)






%%=============================================================================
%% Bijlagen
%%=============================================================================

\appendix
\renewcommand{\chaptername}{Appendix}

%%---------- Onderzoeksvoorstel -----------------------------------------------

\chapter{Onderzoeksvoorstel}

Het onderwerp van deze bachelorproef is gebaseerd op een onderzoeksvoorstel dat vooraf werd beoordeeld door de promotor. Dat voorstel is opgenomen in deze bijlage.

% Verwijzing naar het bestand met de inhoud van het onderzoeksvoorstel
%---------- Inleiding ---------------------------------------------------------

\section{Introductie} % The \section*{} command stops section numbering
\label{sec:introductie}

Iedereen kent wel de term AI. Maar niet iedereen weet dat men dan spreekt over een Artificial Narrow Intelligence. Namelijk een AI die nog steeds beperkt is in wat het kan doen binnen een bepaalde context. Een narrow AI is beperkt tot het doen waarvoor het gemaakt is. Zo kan je bijvoorbeeld aan een helpdesk chatbox geen vragen stellen over een ander bedrijf of een ander domein dan waar deze chatbot voor gemaakt is. \linebreak

Daarentegen bestaat er ook Artificial General Intelligence. En dit is eigenlijk een vorm van AI dat dit alles wel kan. Een AI met de capaciteit om een verscheidenheid aan complexe problemen op te lossen in verschillende domeinen. Dat zichzelf autonoom controleert, met zijn eigen gedachten, zorgen, gevoelens, sterktes, zwaktes en neigingen. \linebreak

Sinds de opkomst van AI is Artificial General Intelligence eigenlijk het oorspronkelijke focus veld van AI. Maar al snel werd duidelijk dat het een zeer complex idee is. Sommigen denken dat het zelfs onmogelijk is om een Artificial General Intelligence te bouwen. Terwijl mensen die meer ervaren zijn in het vak het slechts een technisch probleem vinden. \linebreak

Zeker met de opkomst van AI's zoals ChatGPT door \hfill \break OpenAI, dat een chatbot is die al in een zeer groot aantal domeinen kan antwoorden op vragen, is het interessant om te onderzoeken naar wat juist een AGI is en waarom deze zo moeilijk te bouwen is. \linebreak

In deze bachelorproef zal er een Artificial Narrow Intelligence opgezet worden en zal er onderzocht worden wat de mogelijkheden zijn om deze verder te ontwikkelen naar een Artificial General Intelligence. Hier zal men vooral focussen op onderstaande zaken: \linebreak

\begin{itemize}
    \item waarin verschilt een AGI van een ANI
    \item welke valkuilen zijn er bij het omvormen van een ANI naar een AGI
    \item waarom is het al dan niet mogelijk om een AGI te ontwikkelen
    \item zijn er al pogingen gedaan? En hoe liepen deze?
\end{itemize}

%---------- Stand van zaken ---------------------------------------------------

\section{Literatuurstudie}
\label{sec:state-of-the-art}

In deze literatuurstudie zal er een beeld gevormd worden over wat Artificial Narrow Intelligence en Artificial General Intelligence juist is.

\subsection{Artificial Narrow Intelligence}

De term 'Artificial Intelligence' is voor het eerst benoemd in 1955 door John McCarthy. Sinds de eerste aanraking met AI is er ondertussen al enorm veel dat AI de dag van vandaag kan, en is er nog steeds veel ruimte voor uitbreiding.

Wat veel mensen niet weten is dat wanneer men refereert naar AI of Artificial Intelligence, dat men eigenlijk praat over Artificial Narrow Intelligence, waarom zal verder in deze literatuurstudie besproken worden. 

De grootste groei van AI de afgelopen jaren kan men vinden in enerzijds perceptie en anderzijds cognitie. In perceptie is de grootste evolutie gebeurd op vlak van spraakherkenning. Dit is iets dat door tal van mensen wordt gebruikt, denk bijvoorbeeld maar aan Alexa, Siri,... Hetzelfde geldt voor afbeeldingsherkenning.

Wetende dat bijvoorbeeld spraakherkenning pas sinds de zomer van 2016 enorm is beginnen groeien kan men wel stellen dat evolutie in AI enorm snel kan gaan, dit is ook de reden waarom in deze bachelorproef de evolutie van artificial narrow intelligence naar artificial general intelligence onderzocht zal worden, hier over later meer. \autocite{brynjolfsson2017artificial}

Als we gaan kijken naar wat Artifical Narrow Intelligence juist is, kan men stellen dat dit een AI is die gebouwd is om een bepaalde taak uit te voeren binnen een bepaald domein. Eventuele kennis die deze AI vergaart door deze taak uit te voeren zal niet automatisch worden gebruikt om andere taken uit te voeren buiten dit domein. 

\subsection{Artificial General Intelligence}

In tegenstelling tot Artificial Narrow Intelligence verwijst Artificial General Intelligence naar een AI die even veel kan als de cognitieve systemen van een mens. Hiermee wordt bedoelt dat die verschillende soorten taken, uit verschillende domeinen kan uitvoeren. Door het uitvoeren van deze taken leert de AI bij zoals een mens dit doet en kan hij bijleren om vervolgens taken uit andere domeinen ook te kunnen oplossen. 

Artificial General Intelligence heeft het vermogen om kennis te verwerven en toe te passen, en om te redeneren en te denken, op verschillende gebieden. Je kan stellen dat Artificial General Intelligence streeft naar 'Algemene Intelligentie', zoals dit ook uit de engelse term afgeleid kan worden. 

Uit een studie is gebleken dat wanneer we menselijke algemene intelligentie willen toepassen op een AI, die het volgende moet bevatten: \linebreak

\begin{itemize}
    \item Het vermogen om algemene problemen op een niet-domeingebonden manier op te lossen
    \item Het vermogen om specifieke (taakgebonden) en algemene intelligentie samen te gebruiken
    \item Het vermogen om te leren van zijn omgeving, andere ai's en leraren, al dan niet menselijk
    \item Het vermogen om beter te worden in het oplossen van nieuwe soorten problemen naarmate de ai meer ervaring op doet met dit type problemen
\end{itemize}

Het belangrijkste om te onthouden is dat een belangrijk aspect van Artificial General Intelligence draait om het feit dat een systeem autonoom kan leren, en kennis die het verstrekt heeft door een taak te voltooien kan gebruiken om een andere taak op te lossen, ongeacht welk domein deze taak zich in bevindt. 

Het systeem moet kunnen interageren met zijn omgeving en met andere entiteiten in zijn omgeving, dit kunnen ai's of menselijke interacties zijn. De AI leert bij uit deze interacties. Verder is het in staat om verder te bouwen op eerdere ervaringen en de vaardigheden die daar geleerd zijn te onthouden en toe te passen op complexere taken, om zo ook deze af te ronden. Op deze manier leert een Artificial General Intelligence als maar bij en kan het steeds complexere doelen bereiken.

\autocite{goertzel2007artificial}

% Voor literatuurverwijzingen zijn er twee belangrijke commando's:
% \autocite{KEY} => (Auteur, jaartal) Gebruik dit als de naam van de auteur
%   geen onderdeel is van de zin.
% \textcite{KEY} => Auteur (jaartal)  Gebruik dit als de auteursnaam wel een
%   functie heeft in de zin (bv. ``Uit onderzoek door Doll & Hill (1954) bleek
%   ...'')

%---------- Methodologie ------------------------------------------------------
\section{Methodologie}
\label{sec:methodologie}
Enerzijds zal het onderzoek er uit bestaan om een Artificial Narrow Intelligence op te zetten via het Azure platform, en er zal gekeken worden wat er nodig is om dit op te zetten.

Anderzijds zal er onderzocht worden wat er typerend is aan Artificial General Intelligence en wat hier de valkuilen en opportuniteiten zijn. 

Er zal verder ook bekeken worden of het al dan niet mogelijk is om de opgezette Artificial Narrow Intelligene om te vormen naar een Artificial General Intelligence. 

%---------- Verwachte resultaten ----------------------------------------------
\section{Verwachte resultaten}
\label{sec:verwachte_resultaten}

Bij het opstellen van de Artificial Narrow Intelligence wordt er verwacht dat dit lukt en men kan onderzoeken wat hier \break typisch aan is. Met deze gegevens kan men vergelijken waarin een Artificial General Intelligence hier van verschilt. \linebreak Bij het onderzoeken naar hoe een Artificial General \break Intelligence dient opgezet te worden verwacht men dat het ontwikkelen van AGI een technisch en wetenschappelijk uitdaging is. Het vereist een diepgaande kennis van machine learning, kunstmatige intelligentie, en computerwetenschappen, en het vereist ook een goed begrip van de menselijke geest en de manier waarop we intelligentie definiëren. \linebreak Desalniettemin zal het wel duidelijk worden wat er nodig is en waarom het al dan niet mogelijk is om een AGI te ontwikkelen.
\newpage

%---------- Verwachte conclusies ----------------------------------------------
\section{Verwachte conclusies}
\label{sec:verwachte_conclusies}
Er wordt verwacht dat het opzetten van een Artificial Narrow Intelligence mogelijk zal zijn, en dat er een duidelijke analyse kan gemaakt worden van wat er essentieel is om dit op te zetten. 

Er wordt verwacht dat het onderzoeken wat er essentieel is om een Artificial General Intelligence op te zetten complex zal zijn, maar er hier zeker een duidelijk beeld over kan gevormd worden. Het effectief opzetten (en laten werken) van een Artificial General Intelligence wordt verwacht niet te lukken, wel zal er veel geleerd worden over waarom dit juist niet lukt, en welke evolutie de huidige technologie nog dient te doorstaan alvorens dit wel mogelijk blijkt.



%%---------- Andere bijlagen --------------------------------------------------
% TODO: Voeg hier eventuele andere bijlagen toe
%\input{...}

%%---------- Referentielijst --------------------------------------------------

\printbibliography[heading=bibintoc]

\end{document}
