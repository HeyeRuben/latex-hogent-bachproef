%%=============================================================================
%% Samenvatting
%%=============================================================================

% TODO: De "abstract" of samenvatting is een kernachtige (~ 1 blz. voor een
% thesis) synthese van het document.
%
% Deze aspecten moeten zeker aan bod komen:
% - Context: waarom is dit werk belangrijk?
% - Nood: waarom moest dit onderzocht worden?
% - Taak: wat heb je precies gedaan?
% - Object: wat staat in dit document geschreven?
% - Resultaat: wat was het resultaat?
% - Conclusie: wat is/zijn de belangrijkste conclusie(s)?
% - Perspectief: blijven er nog vragen open die in de toekomst nog kunnen
%    onderzocht worden? Wat is een mogelijk vervolg voor jouw onderzoek?
%
% LET OP! Een samenvatting is GEEN voorwoord!

%%---------- Nederlandse samenvatting -----------------------------------------
%
% TODO: Als je je bachelorproef in het Engels schrijft, moet je eerst een
% Nederlandse samenvatting invoegen. Haal daarvoor onderstaande code uit
% commentaar.
% Wie zijn bachelorproef in het Nederlands schrijft, kan dit negeren, de inhoud
% wordt niet in het document ingevoegd.

\IfLanguageName{english}{%
\selectlanguage{dutch}
\chapter*{Samenvatting}



\selectlanguage{english}
}{}

%%---------- Samenvatting -----------------------------------------------------
% De samenvatting in de hoofdtaal van het document

\chapter*{\IfLanguageName{dutch}{Samenvatting}{Abstract}}

Deze bachelorproef heeft aangetoond dat het combineren van verschillende artificial narrow intelligences (AI's) een veelbelovende aanpak kan zijn om een conversatie-ervaring te creëren die lijkt op een interactie met een persoon. Het onderzoek richtte zich op beeld, spraak en kennis om een realistische en gepersonaliseerde ervaring te bieden aan gebruikers.

Dit onderzoek is belangrijk vanwege de groeiende mogelijkheden en toepassingen van deze technologieën, met name in het ontwikkelen van software die een gesprek kan voeren via videocalls en bijna niet te onderscheiden is van een echt persoon.

Men heeft kennis opgedaan binnen het veld en getracht een overzicht te geven van welke narrow ai's / technologieën kunnen gebruikt worden om dit te bereiken. Voor beeld werden gezichtsherkenning, emotieherkenning en deepfakes gekozen om een gepersonaliseerde visuele weergave te genereren. Op het gebied van spraak werden verschillende AI-technologieën zoals spraakherkenning, spraakverwerking, taalherkenning en taalverwerking gekozen om een natuurlijke en vloeiende conversatie mogelijk te maken. Synthetische stemtechnologie werd gekozen om de interactie nog realistischer te maken. Voor kennis werden algemene kennisverwerving, vaardigheidsverbetering en persoonlijkheidsmodellering besproken, samen met contextueel begrip om een beter begrip van de gebruiker te krijgen en gerichte reacties te bieden.

De resultaten van de enquête bevestigden de belangrijkheid van de gekozen aspecten voor een realistische conversatie-ervaring. Geen enkel onderdeel scoorde lager dan "belangrijk" op de beoordelingsschaal. Hoewel er ruimte is voor verdere ontwikkeling en optimalisatie van de voorgestelde benadering, legt deze bachelorproef een solide basis en biedt het inzicht in de mogelijkheden van AI-technologieën voor conversatie-ervaringen.

Toekomstig onderzoek kan zich richten op de technische implementatie van de voorgestelde benadering en het verfijnen ervan. Een mogelijke vervolgstudie, zoals een masterproef, kan zich concentreren op de daadwerkelijke ontwikkeling en implementatie van de voorgestelde software. Deze bachelorproef biedt waardevolle inzichten voor toekomstig onderzoek op het gebied van AI-gestuurde conversatie-ervaringen, en heeft aangetoond dat het publiek positief reageert op de voorgestelde aspecten en hun belang erkent.