%%=============================================================================
%% Voorwoord
%%=============================================================================

\chapter*{\IfLanguageName{dutch}{Woord vooraf}{Preface}}
\label{ch:voorwoord}

%% TODO:
%% Het voorwoord is het enige deel van de bachelorproef waar je vanuit je
%% eigen standpunt (``ik-vorm'') mag schrijven. Je kan hier bv. motiveren
%% waarom jij het onderwerp wil bespreken.
%% Vergeet ook niet te bedanken wie je geholpen/gesteund/... heeft

De afgelopen jaren heeft AI een enorme ontwikkeling doorgemaakt. Er zijn een groot aantal AI's ontwikkeld die goed zijn in het uitvoeren van specifieke taken, enkele voorbeelden zijn beeldgeneratie, taalvertaling en spraakherkenning. Deze AI's worden ook wel narrow AI's genoemd, omdat ze beperkt zijn in het combineren van taken.

Narrow AI's zijn vooral zeer goed in het uitvoeren van één specifieke taak, en kunnen ze niet echt leren van ervaringen om zich zo aan te passen aan nieuwe situaties. Dit is waar een AGI verschilt van een ANI, omdat een AGI in staat is om te leren en zich aan te passen aan verschillende situaties, net zoals een mens.

In mijn bachelorproef wil ik onderzoeken welke combinaties van ANI's nodig zijn om een indruk te geven dat men te maken heeft met een AGI. Dit is een interessante vraag omdat het kan helpen begrijpen welke zaken essentieel zijn voor het ontwikkelen van een AGI.

Mijn motivatie om deze bachelorproef te schrijven is vooral gebaseerd op de recente evolutie in het domein van AI, aangezien er steeds meer AI's worden ontwikkeld die zaken genereren die vervolgens moeilijk van menselijke creaties te onderscheiden zijn. Door te onderzoeken welke combinaties van ANI's nodig zijn om een AGI te simuleren, hoop ik duidelijkheid te kunnen scheppen over waar de lijn ligt tussen het onderscheiden van contact met een technologie of een mens.

Ik wil graag mijn begeleiders bedanken die mij geholpen hebben tijdens het schrijven van deze bachelorproef.
