%%=============================================================================
%% Voorwoord
%%=============================================================================

\chapter*{\IfLanguageName{dutch}{Woord vooraf}{Preface}}
\label{ch:voorwoord}

%% TODO:
%% Het voorwoord is het enige deel van de bachelorproef waar je vanuit je
%% eigen standpunt (``ik-vorm'') mag schrijven. Je kan hier bv. motiveren
%% waarom jij het onderwerp wil bespreken.
%% Vergeet ook niet te bedanken wie je geholpen/gesteund/... heeft

De afgelopen jaren heeft AI een enorme ontwikkeling doorgemaakt. Er zijn een groot aantal AI's ontwikkeld die goed zijn in het uitvoeren van specifieke taken, enkele voorbeelden zijn beeldgeneratie, taalvertaling en spraakherkenning. Deze AI's worden ook wel narrow AI's genoemd, omdat ze beperkt zijn in het combineren van taken.

De ontwikkeling van artificiele intelligentie en machine learning heeft de afgelopen jaren grote sprongen vooruit gemaakt. Zo hebben slimme assistenten als Siri, Alexa en Google Assistant ons leven drastisch veranderd door ons te helpen met dagelijkse taken, het beantwoorden van vragen en het aansturen van slimme apparaten. Maar wat als we nog verder zouden gaan, en in staat zouden zijn om software te ontwikkelen die het gevoel van een echte conversatie kan nabootsen, op vlak van zowel spraak, beeld als kennis?

In deze bachelorproef neem ik u mee door de verschillende AI-technologieën die gebruikt kunnen worden om dit doel te bereiken. Ik beschrijf hoe deze technologieën werken, hoe ze kunnen worden gecombineerd en hoe ze kunnen worden toegepast om de gebruiker het gevoel te geven dat hij of zij een echte conversatie heeft met een persoonlijke persoon.

Mijn motivatie om deze bachelorproef te schrijven is vooral gebaseerd op de recente evolutie in het domein van AI, aangezien er steeds meer AI's worden ontwikkeld die zaken genereren die vervolgens moeilijk van menselijke creaties te onderscheiden zijn. Door te onderzoeken welke combinaties van ANI's nodig zijn om het bovenstaande te simuleren, hoop ik duidelijkheid te kunnen scheppen over dit thema aan de hand van een concrete case.

Ik hoop dat deze bachelorproef bijdraagt aan het inzicht in de mogelijkheden en beperkingen van narrow AI-technologieën en hoe deze kunnen worden toegepast om een meer menselijke conversatie-ervaring te bieden.

Ik wil graag mijn dank uitspreken aan mijn begeleider en alle andere mensen die mij hebben geholpen bij het uitvoeren van dit onderzoek. Dit omvat niet alleen iedereen die mij hebben voorzien van waardevolle informatie en inzichten, maar ook iedereen die mij heeft gesteund tijdens de soms uitdagende momenten van dit proces.

