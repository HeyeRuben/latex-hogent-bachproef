%%=============================================================================
%% Inleiding
%%=============================================================================

\chapter{\IfLanguageName{dutch}{Inleiding}{Introduction}}
\label{ch:inleiding}

Met de recente ontwikkelingen in AI (GPT) en machine learning wordt het steeds meer mogelijk om software te ontwikkelen die het gevoel van een echte conversatie kan nabootsen, op vlak van zowel spraak, beeld als kennis. Dit opent de deur naar nieuwe en meer geavanceerde toepassingen, zoals het ontwikkelen van een software die een gesprek kan hebben met een persoon via een videocall en die nauwelijks van een echt persoon te onderscheiden is.

In deze bachelorproef wordt onderzocht welke combinaties van narrow AI's nodig zijn om dit doel te bereiken. Er wordt beschreven hoe verschillende AI-technologieën werken, hoe ze kunnen worden gecombineerd en hoe ze kunnen worden toegepast om de gebruiker het gevoel te geven dat hij of zij een echte conversatie heeft met een echt persoon. Dit onderzoek is niet alleen relevant voor developers die deze software zouden willen bouwen, maar ook voor gebruikers van het internet, omdat het hen inzicht kan geven in wat er mogelijk is en een waarschuwing kan geven dat niet elke interactie die je online hebt sowieso met een persoon is, maar ook met een AI kan zijn.

Door deze ontwikkelingen wordt het steeds belangrijker om te begrijpen hoe narrow AI's kunnen worden gecombineerd om de ervaring van menselijke conversatie na te bootsen. Deze bachelorproef hoopt bij te dragen aan een beter inzicht in de mogelijkheden van deze technologieën en hoe ze kunnen worden toegepast om een meer menselijke conversatie-ervaring te bieden.

\section{\IfLanguageName{dutch}{Onderzoeksdoelstelling}{Research objective}}
\label{sec:onderzoeksdoelstelling}

De doelstelling van dit onderzoek is om een overzicht te geven van enkele narrow AI-technologieën die nodig zouden zijn om applicatie te ontwikkelen dat in staat is om een realistische conversatie-ervaring te bieden die niet van menselijke interactie te onderscheiden is. Hierbij zal er onderzocht worden welke combinaties van AI's hiervoor geschikt zouden kunnen zijn en hoe deze gecombineerd kunnen worden om dit doel te bereiken.

Daarnaast zal er een enquête worden gehouden om de verschillende menselijke aspecten van de case te evalueren op belangrijkheid. 
Hierbij zal het publiek gevraagd worden om bij verschillende aspecten van de conversatie-ervaring zoals bijvoorbeeld op vlak van beeld: de gegeneerde persoon, of de gelaatsuitdrukkingen en hiervan de belangrijkheid te scoren. Deze gegevens zullen worden geanalyseerd om te bepalen welke aspecten het belangrijkste zijn voor het creëren van een realistische conversatie-ervaring met een combinatie van AI-systemen.

\section{\IfLanguageName{dutch}{Opzet van deze bachelorproef}{Structure of this bachelor thesis}}
\label{sec:opzet-bachelorproef}

% Het is gebruikelijk aan het einde van de inleiding een overzicht te
% geven van de opbouw van de rest van de tekst. Deze sectie bevat al een aanzet
% die je kan aanvullen/aanpassen in functie van je eigen tekst.

De rest van deze bachelorproef is als volgt opgebouwd:

In Hoofdstuk~\ref{ch:literatuurstudie} wordt een overzicht gegeven van de stand van zaken binnen het onderzoeksdomein, op basis van een literatuurstudie.

In Hoofdstuk~\ref{ch:methodologie} wordt de methodologie toegelicht en worden de gebruikte onderzoekstechnieken besproken om een antwoord te kunnen formuleren op de onderzoeksvragen.

% TODO: Vul hier aan voor je eigen hoofstukken, één of twee zinnen per hoofdstuk

In Hoofdstuk~\ref{ch:conclusie}, tenslotte, wordt de conclusie gegeven en een antwoord geformuleerd op de onderzoeksvragen. Daarbij wordt ook een aanzet gegeven voor toekomstig onderzoek binnen dit domein.