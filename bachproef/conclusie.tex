%%=============================================================================
%% Conclusie
%%=============================================================================

\chapter{Conclusie}
\label{ch:conclusie}

% TODO: Trek een duidelijke conclusie, in de vorm van een antwoord op de
% onderzoeksvra(a)g(en). Wat was jouw bijdrage aan het onderzoeksdomein en
% hoe biedt dit meerwaarde aan het vakgebied/doelgroep? 
% Reflecteer kritisch over het resultaat. In Engelse teksten wordt deze sectie
% ``Discussion'' genoemd. Had je deze uitkomst verwacht? Zijn er zaken die nog
% niet duidelijk zijn?
% Heeft het onderzoek geleid tot nieuwe vragen die uitnodigen tot verder 
%onderzoek?

\section{Onderzoek naar combinaties van artificial narrow intelligences om menselijke aspecten na te bootsen}

Deze bachelorproef heeft aangetoond dat het combineren van verschillende artificial narrow intelligences een potentieel succesvolle aanpak kan zijn om een conversatie-ervaring te creëren die de indruk geeft dat men met een persoon praat.

Bij beeld werd er gekozen voor gezichtsherkenning en emotieherkenning om een meer gepersonaliseerde ervaring te bieden aan de gebruiker. Deepfakes zijn dan de logische keuze om een beeld te genereren van een persoon die niet bestaat.

Voor spraak werd er voor verschillende AI-technologieën gekozen zoals spraakherkenning, spraakverwerking, taalherkenning en taalverwerking om de gebruiker een natuurlijke en vloeiende conversatie-ervaring te bieden. Synthetische stemtechnologie kan gebruikt worden om de interactie nog realistischer te maken.

Bij kennis werd gekozen voor het verwerven en verbeteren van algemene kennis en vaardigheden. Ook persoonlijkheidsmodellering om de ervaring verder te personaliseren. Daarnaast werd contextueel begrip toegevoegd om de gebruiker beter te begrijpen en een meer gerichte respons te kunnen geven.

Een combinatie van deze verschillende AI-technologieën kan ons in staat stellen om een realistische en gepersonaliseerde conversatie-ervaring te schetsen, die vergelijkbaar is met een menselijke interactie. Het kan echter nog verder worden ontwikkeld en geoptimaliseerd om een nog betere gebruikerservaring te bieden.

Een eventueel verder onderzoek kan duidelijk scheppen over hoe men deze narrow ai's ook effectief technisch gaat combineren. Ook eventueel een onderzoek naar het effectief opzetten van deze hele case kan een mooi onderzoek zijn voor bijvoorbeeld een masterproef.

\section{Bevraging naar de belangrijkheid van deze aspecten}

De bevraging heeft bijgedragen aan het begrijpen van de belangrijkheid van verschillende menselijke aspecten in een conversatie-ervaring. 

De resultaten van de enquête bevestigde dat de gekozen aspecten die deel zouden uitmaken van de applicatie om een conversatie ervaring na te bootsen ook effectief belangrijk waren voor de populatie. Zo is er geen enkel onderdeel dat afgerond minder scoorde dan een 4 (Belangrijk).

De vergelijkingen tussen de twee leeftijdsgroepen en de vergelijkingen tussen mensen zonder kennis van AI en mensen met kennis van AI vertoonden slechts kleine verschillen. Het feit dat de enquête vragen stelde over het belang van menselijke aspecten kan mogelijk verklaren waarom deze verschillen niet significant waren. Uiteindelijk blijkt dat het hebben van een perceptie van een echt menselijk persoon niet sterk afhankelijk is van leeftijd of kennis van AI.

\section{Verder onderzoek}

Verdere onderzoeken voor deze bachelorproef kunnen zich richten op het verfijnen en optimaliseren van de voorgestelde benadering voor het creëren van een realistische en gepersonaliseerde conversatie-ervaring. Hoewel de technische implementatie niet is behandeld in deze bachelorproef, biedt het een solide basis en richting voor toekomstig onderzoek.

Deze bachelorproef biedt een waardevolle basis voor een toekomstige masterproef die gericht is op het daadwerkelijk implementeren van de voorgestelde software. Het geeft inzicht in de verschillende AI-technologieën die kunnen worden gecombineerd om een conversatie-ervaring te creëren die lijkt op een interactie met een persoon. Bovendien is uit de enquête gebleken dat het publiek positief heeft gereageerd op de voorgestelde aspecten en de belangrijkheid ervan heeft erkend.



