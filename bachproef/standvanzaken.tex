\chapter{\IfLanguageName{dutch}{Stand van zaken}{State of the art}}
\label{ch:stand-van-zaken}

% Tip: Begin elk hoofdstuk met een paragraaf inleiding die beschrijft hoe
% dit hoofdstuk past binnen het geheel van de bachelorproef. Geef in het
% bijzonder aan wat de link is met het vorige en volgende hoofdstuk.

% Pas na deze inleidende paragraaf komt de eerste sectiehoofding.

Voor we het onderzoek kunnen voeren naar hoe we kunnen evolueren van een Artificial Narrow Intelligence naar een Artificial General Intelligence is het eerst belangrijk om grondig uit te leggen wat beide termen juist inhouden en welke onderzoeken er al zijn uitgevoerd in het domein dat we onderzoeken.

Eerst zal men de geschiedenis van AI even schetsen, en benadrukken waarom de evolutie van ANI naar AGI misschien niet zo ver in de toekomst ligt, zoals veel experts denken.

\section{Waar staat AI vandaag, en hoe snel gaat de evolutie?}
De term 'AI' is een term die we in het huidige tijdperk niet meer kunnen wegdenken. Het is alom bekend en wordt in veel (grote) bedrijven gebruikt voor allerhande toepassingen. Maar hoe zijn we hier geraakt? En is deze evolutie verlopen zoals we pakweg 20 jaar geleden zouden verwacht hebben?

\subsection{De populariteit van AI}
Volgens \cite{brynjolfsson2017artificial} dat zich vooral focust op AI in de bedrijfswereld is AI de dag van vandaag al enorm aanwezig in een enorm aantal bedrijven. Dit komt omdat er voortdurend 'general purpose' technologieën opduiken, dit zijn technologieën die de mogelijkheid hebben om een hele economie te kunnen beïnvloeden en het potentieel hebben om samenlevingen drastisch te veranderen door hun impact op al bestaande sociale en economische structuren. Neem nu bijvoorbeeld de verbrandingsmotor die er voor zorgde dat auto's, truck's,... gecommercialiseerd konden worden door bijvoorbeeld bedrijven zoals UPS of Uber. De meest belangrijke 'general purpose' technologie van deze tijd is machine learning: dit is het vermogen van een machine om te kunnen blijven evolueren en het voortdurend verbeteren van zijn prestaties zonder dat mensen exact moeten uitleggen hoe ze een bepaalde taak moeten volbrengen die hen gegeven wordt. 

Dit is dan ook deels de grootste factor waarom AI de dag van vandaag zo populair is, omdat er economisch, en dus binnen de bedrijfswereld ook enorm veel vooruitgang kan geboekt worden. Zeker de afgelopen jaren is er enorme vooruitgang geboekt om machines taken autonoom te laten uitvoeren.

\subsection{Wat kan AI momenteel al?}

AI, of Artificiële intelligentie wordt meestal gezien als de capaciteit van machines om menselijke taken uit te voeren, voornamelijk zaken die te maken hebben met congnitieve functies zoals luisteren, kijken, spreken,... 

Volgens enquetes van \cite{benbya2020artificial} wordt er gesuggereerd dat minder dan de helft van de organisaties zinvolle AI projecten hebben lopen, of het vooruitzicht hebben dat deze zullen plaatsvinden, dus hier is nog veel ruimte voor groei. 

Maar wat is het dan juist? Waar is AI nuttig voor, en waarvoor wordt het gebruikt? Dit zal hier verder besproken worden.

\subsubsection{Business}
Ook al blijven de meeste AI-projecten die opgezet worden in de bedrijfswereld experimenteel, of een proof of concept, is het niet zo dat er geen enkel bedrijf is waar AI gebruikt wordt. De grootste reden dat bedrijven hun AI-project niet uitrollen heeft te maken met de weinig opbrengsten die hiermee behaald konden worden.

Toch wordt AI effectief gebruikt en kan men uit een enquete, afgenomen door Deloitte, waar bedrijfsleiders werden gevraagd waarvoor ze AI gebruiken, het onderstaande afleiden (percentages betekenen hoe vaak de antwoorden zijn voorgekomen bv. 'om keuzes maken te vergemakkelijken' kwam voor in 1 op 4 van de enquetes, er konden tevens ook meerdere antwoorden worden geselecteerd)

\begin{itemize}
    \item 28\% om processen te vergemakkelijken
    \item 25\% om bestaande producten of services te verbeteren
    \item 23\% om nieuwe producten of services te creëren
    \item 21\% om keuzes maken te vergemakkelijken
    \item 20\% om kosten te verlagen
\end{itemize}

Een interessant gegeven is dat AI vaak genoemd wordt om het aantal werknemers te verminderen in een bedrijf, werd dit maar in 11\% van de enquetes genoemd. 

Echter in deze tendens wordt er nu een verschuiving gemerkt. Waar werkgevers initieel enkel focuste op het gebruiken van AI om bepaalde specifieke workflows, processen en repetitief werk te automatiseren, wordt er nu meer gekeken om AI in te zetten voor niet systematische congitieve taken, die zelfs keuzes kunnen maken en problemen kunnen oplossen. Zelfs creativiteit is iets waarvoor AI momenteel kan gebruikt worden. Dit is iets wat pakweg 5 jaar geleden buiten de scope van AI viel.

Hoe en waarom deze tendens gebroken werd zal later in deze literatuurstudie aan bod komen. 

Eerst is het belangrijk om een goed beeld te vormen van wat er momenteel (technologisch) mogelijk is met AI.

\subsubsection{Technologie}
Het is belangrijk om een beeld te scheppen van de evolutie van de technologie achter AI tot dusver.

Eerst is het belangrijk om te benadrukken dat de term 'AI' redelijk breed is.

Uit de studie van \cite{benbya2020artificial} kan men AI types bekijken van uit 3 inzichten:
\begin{itemize}
    \item Gebasseerd op functie
    \item Gebasseerd op technologie
    \item Gebasseerd op intelligentie
\end{itemize}

We kunnen deze 3 brede takken nog iets beter toelichten per tak.

Gebasseerd op functie: 

    Hier wordt het onderscheid gemaakt tussen vier soorten artificiële intelligentie. Namelijk conversationale, biometrische, algoritmische en robotische AI. Hier gaat het dus louter om de functie waarvoor de AI gebruikt wordt. 
    
    Zo is conversationele AI sinds de opkomst van OpenAI en ChatGPT zeer populair, dit is zoals het woord het wel al doet vermoeden, een ai die in staat is om menselijke taal te herkennen, en te begrijpen. Dit doormiddel van tekst -en stemherkenning. Conversationele AI heeft bijgevolg het meeste kans om meer complexe taken te kunnen uitvoeren, omdat het duidelijker kan gemaakt worden wat de opdracht is, als de AI het taalmodel begrijpt van de opdrachtgever, ook opnieuw is hier het beste voorbeeld de GPT4 technologie. Hier over later meer. 
    
    Biometrische AI gaat dan weer om met het fysiologische aspect van de mens en heeft als functie bijvoorbeeld vingerafdrukken herkennen, iris scanner,... maar ook bijvoorbeeld het herkennen van gedragskenmerken zoals een handtekening, stem,... 
    
    Algoritmische AI heeft voornamelijk te maken met machine learning (ML) algoritmen. Dit zijn een aantal instructies die een computer kan uitvoeren. Dit kan bijvoorbeeld zijn om een bepaalde specifieke taak aan te leren aan de AI.

    Robotische AI slaat op fysieke robots, deze worden al geringe tijd gebruikt in bijvoorbeeld fabrieken. Robots met AI kunnen hun omgeving waarnemen, begrijpen, leren en hier acties op ondernemen. Dit draagt bij aan het feit dat robots een groot aantal taken kunnen uitvoeren, dit kan gaan over mensen assistentie bieden bij een aantal taken, of het identificeren van objecten, en op basis van deze objecten bepaalde taken uitvoeren.

Gebasseerd op technologie:

    Hier is het onderscheid gebasseerd op de technologie die gebruikt wordt in het AI systeem. Dit gaat vooral over machine learning, deep learning, neural networks, natural language processing, rule-based expert systemen, robotic process automation en robots.
    
    Machine learning kan men opzich nog eens opdelen in 3 takken, namelijk reinforcement learning, supervised learning en unsupervised learning.
    Kort uitgelegd: Reinforced learning slaat op een AI dat kan leren uit ervaring, zonder menselijke invloed. Supervised learning is wanneer een AI leert van een dataset waarmee het getrained wordt. Unsupervised learning is wanneer een AI patronen kan herkennen in data dat niet gelabeled is, en waarvan het resultaat nog niet gekend is.
    
    Deep learning is een klasse van machine learning waarbij de AI kan leren zonder supervisie van een persoon. Het kan zichzelf trainen via zowel gelabelde als ongelabelde data. Een welgekend voorbeeld hier van is spraak -en afbeeldingsherkenning.
    
    Neural networks zijn algoritmes die proberen onderliggende relaties uit een dataset te halen via een proces dat de werking van het menselijk brein probeert na te bootsen.
    
    Natural language processing kwam daarnet al aan bod en slaat op het feit dat een AI de mogelijkheid heeft om een taal te herkennen en te begrijpen. bv. ChatGPT
    
    Rule based expert systems zijn systemen waarbij er een aantal gepredefineerde regels worden opgesteld door een persoon. Deze worden dan gevolgd door deze AI, vandaar de naam Rule based.
    
    Robotic process automation systemen zijn systemen die gebouwd zijn om bepaalde digitale taken te automatiseren, dit kan bijvoorbeeld het inwisselen van een credit kaart zijn, waar er een aantal processen dienen uitgevoerd worden, deze gebeuren dan via de AI.
    
    Robots kwamen daarnet ook al aan bod, en slaat op het automatisch bedienen van machines. Bijvoorbeeld in een fabriek.
    
Gebasseerd op intelligentie:

    Dit slaat op de intelligentie van de AI. En dit kan men opdelen in 3 delen, hiervan zijn 2 delen ook tevens de belangrijkste voor deze studie.
    \begin{itemize}
        \item Artificial Narrow Intelligence
        \item Artificial General Intelligence
        \item Artificial Super Intelligence
    \end{itemize}

    Artificial Narrow Intelligence (ANI) en Artificial General Intelligence (AGI) worden verder in deze literatuurstudie besproken, dus hier zal men momenteel niet te diep op in gaan. 
    
    Artificial Narrow Intelligence slaat op een AI dat vaak maar de mogelijkheid heeft om één bepaald probleem op te lossen. En niet in staat is om vaardigheden die het kent om dit bepaald probleem op te lossen, over te dragen naar een ander domein. 
    
    Artificial General Intelligence slaat op een AI die streeft naar het leren van zaken op menselijk niveau. Het kan vaardigheden van het ene domein naar het andere domein overdragen en kan ook linken leggen tussen bepaalde oplossingen voor gegeven problemen. Het kan leren zoals een mens dit kan.
    
    Artificial Super Intelligence is een theoretische vorm van een AI, waar het mogelijk zou zijn om de mensenlijke aspecten te overstijgen. De theorie zegt dat wanneer men er in slaagt om een AGI te ontwikkelen dat deze super intelligence zonder problemen slimmer, beter en meer capaciteiten heeft dan de mens. 


\subsection{Recente veranderingen in het domein}

\subsubsection{ChatGPT}

\subsubsection{Komt dit in de buurt van AGI?}

\section{Artificial Narrow Intelligence}

\subsection{Technische aspecten}

\subsection{Opzetten van een ANI}

\section{Artificial General Intelligence}

\subsection{Technische aspecten}

\subsection{Menselijke aspecten}



