\chapter{\IfLanguageName{dutch}{Stand van zaken}{State of the art}}
\label{ch:stand-van-zaken}

% Tip: Begin elk hoofdstuk met een paragraaf inleiding die beschrijft hoe
% dit hoofdstuk past binnen het geheel van de bachelorproef. Geef in het
% bijzonder aan wat de link is met het vorige en volgende hoofdstuk.

% Pas na deze inleidende paragraaf komt de eerste sectiehoofding.

Voor we het onderzoek kunnen voeren naar hoe we kunnen evolueren van een Artificial Narrow Intelligence naar een Artificial General Intelligence is het eerst belangrijk om grondig uit te leggen wat beide termen juist inhouden en welke onderzoeken er al zijn uitgevoerd in het domein dat we onderzoeken.

Eerst zal men de geschiedenis van AI even schetsen, en benadrukken waarom de evolutie van ANI naar AGI misschien niet zo ver in de toekomst ligt, zoals veel experts denken.

\section{Waar staat AI vandaag, en hoe snel gaat de evolutie?}
De term 'AI' is een term die we in het huidige tijdperk niet meer kunnen wegdenken. Het is alom bekend en wordt in veel (grote) bedrijven gebruikt voor allerhande toepassingen. Maar hoe zijn we hier geraakt? En is deze evolutie verlopen zoals we pakweg 20 jaar geleden zouden verwacht hebben?

\subsection{De populariteit van AI}
Volgens \cite{brynjolfsson2017artificial} dat zich vooral focust op AI in de bedrijfswereld is AI de dag van vandaag al enorm aanwezig in een enorm aantal bedrijven. Dit komt omdat er voortdurend 'general purpose' technologieën opduiken, dit zijn technologieën die de mogelijkheid hebben om een hele economie te kunnen beïnvloeden en het potentieel hebben om samenlevingen drastisch te veranderen door hun impact op al bestaande sociale en economische structuren. Neem nu bijvoorbeeld de verbrandingsmotor die er voor zorgde dat auto's, truck's,... gecommercialiseerd konden worden door bijvoorbeeld bedrijven zoals UPS of Uber. De meest belangrijke 'general purpose' technologie van deze tijd is machine learning: dit is het vermogen van een machine om te kunnen blijven evolueren en het voortdurend verbeteren van zijn prestaties zonder dat mensen exact moeten uitleggen hoe ze een bepaalde taak moeten volbrengen die hen gegeven wordt. 

Dit is dan ook deels de grootste factor waarom AI de dag van vandaag zo populair is, omdat er economisch, en dus binnen de bedrijfswereld ook enorm veel vooruitgang kan geboekt worden. Zeker de afgelopen jaren is er enorme vooruitgang geboekt om machines taken autonoom te laten uitvoeren.

\subsection{Wat kan AI momenteel al?}

AI, of Artificiële intelligentie wordt meestal gezien als de capaciteit van machines om menselijke taken uit te voeren, voornamelijk zaken die te maken hebben met congnitieve functies zoals luisteren, kijken, spreken,... 

Volgens enquetes van \cite{benbya2020artificial} wordt er gesuggereerd dat minder dan de helft van de organisaties zinvolle AI projecten hebben lopen, of het vooruitzicht hebben dat deze zullen plaatsvinden, dus hier is nog veel ruimte voor groei. 

Maar wat is het dan juist? Waar is AI nuttig voor? Dit zal hier verder besproken worden.

\subsubsection{Business}

\subsubsection{Technologie}

\subsection{Recente veranderingen in het domein}

\subsubsection{ChatGPT}

\subsubsection{Komt dit in de buurt van AGI?}

\section{Artificial Narrow Intelligence}

\subsection{Technische aspecten}

\subsection{Opzetten van een ANI}

\section{Artificial General Intelligence}

\subsection{Technische aspecten}

\subsection{Menselijke aspecten}



