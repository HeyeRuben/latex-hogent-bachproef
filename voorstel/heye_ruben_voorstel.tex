%==============================================================================
% Sjabloon onderzoeksvoorstel bachelorproef
%==============================================================================
% Gebaseerd op LaTeX-sjabloon ‘Stylish Article’ (zie voorstel.cls)
% Auteur: Jens Buysse, Bert Van Vreckem
%
% Compileren in TeXstudio:
%
% - Zorg dat Biber de bibliografie compileert (en niet Biblatex)
%   Options > Configure > Build > Default Bibliography Tool: "txs:///biber"
% - F5 om te compileren en het resultaat te bekijken.
% - Als de bibliografie niet zichtbaar is, probeer dan F5 - F8 - F5
%   Met F8 compileer je de bibliografie apart.
%
% Als je JabRef gebruikt voor het bijhouden van de bibliografie, zorg dan
% dat je in ``biblatex''-modus opslaat: File > Switch to BibLaTeX mode.

\documentclass{hogent-article}

\usepackage{lipsum}

%------------------------------------------------------------------------------
% Metadata over het voorstel
%------------------------------------------------------------------------------

%---------- Titel & auteur ----------------------------------------------------

% TODO: geef werktitel van je eigen voorstel op
\PaperTitle{De evolutie van Artificial Narrow Intelligence naar Artificial General Intelligence}
\PaperType{Onderzoeksvoorstel Bachelorproef 2022-2023} % Type document

% TODO: vul je eigen naam in als auteur, geef ook je emailadres mee!
\Authors{Ruben Heye\textsuperscript{1}} % Authors
\CoPromotor{Robin Menschaert\textsuperscript{2} (Inetum-RealDolmen)}
\affiliation{\textbf{Contact:}
  \textsuperscript{1} \href{mailto:ruben.heye@student.hogent.be}{ruben.heye@student.hogent.be};
  \textsuperscript{2} \href{mailto:robin.menschaert@inetum-realdolmen.world}{robin.menschaert@inetum-realdolmen.world};
}

%---------- Abstract ----------------------------------------------------------

\Abstract{Er zijn verschillende redenen waarom onderzoek naar Artificial General Intelligence (AGI) belangrijk is. AGI is een veelbelovende technologie die in staat is om complexe problemen op te lossen in verschillende domeinen. Dit kan leiden tot een enorme vooruitgang op veel gebieden, zoals gezondheid, wetenschap, technologie en economie. AGI kan ons helpen om beter te begrijpen hoe de menselijke geest werkt en hoe we onze eigen intelligentie kunnen verbeteren. Door het ontwikkelen van AGI kunnen we ook leren hoe we machine-learning-systemen beter kunnen ontwerpen en trainen. In deze bachelorproef zal er onderzocht worden hoe moeilijk het is om zowel een Artificial Narrow Intelligence als een Artificial General Intelligence op te zetten, en deze beide te vergelijken. In dit document wordt eerst en vooral kort beschreven wat artificial narrow intelligence, en zijn technologische opvolger, artificial general intelligence juist is. Na dit kort te hebben toegelicht wordt er besproken wat er in deze bachelorproef juist onderzocht zal worden. Het onderzoek zal er uit bestaan om een Artificial Narrow Intelligence op te zetten via het Azure platform, en hier te analyseren wat nodig is. Verder zal er onderzocht worden wat de uitdagingen zijn bij het bouwen van een Artificial General Intelligence en wat de valkuilen en opportuniteiten zijn. Er wordt verwacht dat het duidelijk zal worden wat er nodig is om zowel een ANI als een AGI op te zetten en waarin deze verschillen. Er wordt verwacht dat het onderzoeken wat er essentieel is om een Artificial General Intelligence op te zetten complex zal zijn, maar er hier zeker een duidelijk beeld over kan gevormd worden.
}

%---------- Onderzoeksdomein en sleutelwoorden --------------------------------
% TODO: Sleutelwoorden:
%
% Het eerste sleutelwoord beschrijft het onderzoeksdomein. Je kan kiezen uit
% deze lijst:
%
% - Mobiele applicatieontwikkeling
% - Webapplicatieontwikkeling
% - Applicatieontwikkeling (andere)
% - Systeembeheer
% - Netwerkbeheer
% - Mainframe
% - E-business
% - Databanken en big data
% - Machineleertechnieken en kunstmatige intelligentie
% - Andere (specifieer)
%
% De andere sleutelwoorden zijn vrij te kiezen

\Keywords{Machineleertechnieken en kunstmatige intelligentie --- Artificial Narrow Intelligence --- Artificial General Intelligence} % Keywords
\newcommand{\keywordname}{Sleutelwoorden} % Defines the keywords heading name

%---------- Titel, inhoud -----------------------------------------------------

\begin{document}

\flushbottom % Makes all text pages the same height
\maketitle % Print the title and abstract box
\tableofcontents % Print the contents section
\thispagestyle{empty} % Removes page numbering from the first page

%------------------------------------------------------------------------------
% Hoofdtekst
%------------------------------------------------------------------------------

% De hoofdtekst van het voorstel zit in een apart bestand, zodat het makkelijk
% kan opgenomen worden in de bijlagen van de bachelorproef zelf.
%---------- Inleiding ---------------------------------------------------------

\section{Introductie} % The \section*{} command stops section numbering
\label{sec:introductie}

Met de recente ontwikkelingen in AI (GPT) en machine learning wordt het steeds meer mogelijk om software te ontwikkelen die het gevoel van een echte conversatie kan nabootsen, op vlak van zowel spraak, beeld als kennis. Dit opent de deur naar nieuwe en meer geavanceerde toepassingen, zoals het ontwikkelen van een software die een gesprek kan hebben met een persoon via een videocall en die nauwelijks van een echt persoon te onderscheiden is.

In deze bachelorproef wil men onderzoeken welke combinaties van narrow AI's nodig zijn om dit doel te bereiken. Er zal beschreven worden hoe verschillende AI-technologieën werken, hoe ze kunnen worden gecombineerd en hoe ze kunnen worden toegepast om de gebruiker het gevoel te geven dat hij of zij een echte conversatie heeft met een echt persoon. Dit onderzoek is niet alleen relevant voor developers die deze software zouden willen bouwen, maar ook voor gebruikers van het internet, omdat het hen inzicht kan geven in wat er mogelijk is en een waarschuwing kan geven dat niet elke interactie die je online hebt sowieso met een persoon is, maar ook met een AI kan zijn.

Door deze ontwikkelingen wordt het steeds belangrijker om te begrijpen hoe narrow AI's kunnen worden gecombineerd om de ervaring van menselijke conversatie na te bootsen. Deze bachelorproef hoopt bij te dragen aan een beter inzicht in de mogelijkheden van deze technologieën en hoe ze kunnen worden toegepast om een meer menselijke conversatie-ervaring te bieden.

\begin{itemize}
    \item Welke narrow ai's dient men te combineren om bepaalde menselijke aspecten na te bootsten
    \item Belangrijkheid van deze aspecten bevragen
\end{itemize}

%---------- Stand van zaken ---------------------------------------------------

\section{Literatuurstudie}
\label{sec:state-of-the-art}

In deze literatuurstudie zal er een beeld gevormd worden over wat Artificial Narrow Intelligence en Artificial General Intelligence juist is.

\subsection{Artificial Narrow Intelligence}

De term 'Artificial Intelligence' is voor het eerst benoemd in 1955 door John McCarthy. Sinds de eerste aanraking met AI is er ondertussen al enorm veel dat AI de dag van vandaag kan, en is er nog steeds veel ruimte voor uitbreiding.

Wat veel mensen niet weten is dat wanneer men refereert naar AI of Artificial Intelligence, dat men eigenlijk praat over Artificial Narrow Intelligence, waarom zal verder in deze literatuurstudie besproken worden. 

De grootste groei van AI de afgelopen jaren kan men vinden in enerzijds perceptie en anderzijds cognitie. In perceptie is de grootste evolutie gebeurd op vlak van spraakherkenning. Dit is iets dat door tal van mensen wordt gebruikt, denk bijvoorbeeld maar aan Alexa, Siri,... Hetzelfde geldt voor afbeeldingsherkenning.

Wetende dat bijvoorbeeld spraakherkenning pas sinds de zomer van 2016 enorm is beginnen groeien kan men wel stellen dat evolutie in AI enorm snel kan gaan, dit is ook de reden waarom in deze bachelorproef de evolutie van artificial narrow intelligence naar artificial general intelligence onderzocht zal worden, hier over later meer. \autocite{brynjolfsson2017artificial}

Als we gaan kijken naar wat Artifical Narrow Intelligence juist is, kan men stellen dat dit een AI is die gebouwd is om een bepaalde taak uit te voeren binnen een bepaald domein. Eventuele kennis die deze AI vergaart door deze taak uit te voeren zal niet automatisch worden gebruikt om andere taken uit te voeren buiten dit domein. 

Wanneer we ons iets meer gaan toespitsen op AI binnen bedrijven, dan merkt men nog steeds dat het gaat om narrow AI’s. Deze worden gebruikt om bepaalde cognitieve (menselijke) taken uit te voeren en te automatiseren. Dit zijn vaak physische processen zoals het bewegen van objecten, voelen, waarnemen, problem solving, keuzes maken en innoveren. 

Bij de meeste organisaties is het gebruiken van AI binnen het bedrijf nog steeds experimenteel. En dit komt deels door de beperking van een Narrow Artificial Intelligence dat zich vaak enkel toespitst op een enkele taak.

Narrow Artificial Intelligence is dus nog steeds maar in staat om één specifiek probleem op te lossen en niet in staat vaardigheden van het ene domein naar het andere over te dragen. Echter nog niet zo lang geleden beginnen AI-programma’s zoals bijvoorbeeld de GPT-3 taalvoorspelling enkele aspecten van algemene intelligentie te vertonen.

\autocite{benbya2020artificial}

Deze aspecten worden besproken in het volgende deel van de litartuurstudie.

\subsection{Artificial General Intelligence}

In tegenstelling tot Artificial Narrow Intelligence verwijst Artificial General Intelligence naar een AI die even veel kan als de cognitieve systemen van een mens. Hiermee wordt bedoelt dat die verschillende soorten taken, uit verschillende domeinen kan uitvoeren. Door het uitvoeren van deze taken leert de AI bij zoals een mens dit doet en kan hij bijleren om vervolgens taken uit andere domeinen ook te kunnen oplossen. 

Artificial General Intelligence heeft het vermogen om kennis te verwerven en toe te passen, en om te redeneren en te denken, op verschillende gebieden. Je kan stellen dat Artificial General Intelligence streeft naar 'Algemene Intelligentie', zoals dit ook uit de engelse term afgeleid kan worden. 

Uit een studie is gebleken dat wanneer we menselijke algemene intelligentie willen toepassen op een AI, die het volgende moet bevatten: \linebreak

\begin{itemize}
    \item Het vermogen om algemene problemen op een niet-domeingebonden manier op te lossen
    \item Het vermogen om specifieke (taakgebonden) en algemene intelligentie samen te gebruiken
    \item Het vermogen om te leren van zijn omgeving, andere ai's en leraren, al dan niet menselijk
    \item Het vermogen om beter te worden in het oplossen van nieuwe soorten problemen naarmate de ai meer ervaring op doet met dit type problemen
\end{itemize}

Het belangrijkste om te onthouden is dat een belangrijk aspect van Artificial General Intelligence draait om het feit dat een systeem autonoom kan leren, en kennis die het verstrekt heeft door een taak te voltooien kan gebruiken om een andere taak op te lossen, ongeacht welk domein deze taak zich in bevindt. 

Het systeem moet kunnen interageren met zijn omgeving en met andere entiteiten in zijn omgeving, dit kunnen ai's of menselijke interacties zijn. De AI leert bij uit deze interacties. Verder is het in staat om verder te bouwen op eerdere ervaringen en de vaardigheden die daar geleerd zijn te onthouden en toe te passen op complexere taken, om zo ook deze af te ronden. Op deze manier leert een Artificial General Intelligence als maar bij en kan het steeds complexere doelen bereiken.

\autocite{goertzel2007artificial}

Momenteel komen de huidige AI systemen nog niet volledig in de buurt van de bovenstaande aspecten. Echter is het wel zo dat er in bepaalde gebieden aanwijzingen zijn die duiden op vooruitgang. Een goed voorbeeld daarvan zijn neurale netwerken, deze vormen de basis van een hele boel recente successen in de AI-wereld. Maar ook zij hebben het nog moeilijk om over te schakelen van een taak uit een specifiek domein naar een taak uit een ander domein. Zo is het gemakkelijk om een specifiek geval van een probleem te nemen en zich te richten op het verbeteren van de prestaties in plaats van te proberen systemen te bouwen met echt diverse vaardigheden. Zo kan men pas enkel AGI bereiken wanneer het systeem effectief uit zichzelf kan leren uit enkele voorbeelden over een enorm spectrum van domeinen.

Hoe men dit eventueel kan bereiken, en wat men dus allemaal nodig blijkt te hebben om dit na te bootsen aan de hand van verschillende Artificial Narrow Intelligences zal onderzocht worden in deze bachelorproef.

\autocite{shevlin2019limits}

\newpage

% Voor literatuurverwijzingen zijn er twee belangrijke commando's:
% \autocite{KEY} => (Auteur, jaartal) Gebruik dit als de naam van de auteur
%   geen onderdeel is van de zin.
% \textcite{KEY} => Auteur (jaartal)  Gebruik dit als de auteursnaam wel een
%   functie heeft in de zin (bv. ``Uit onderzoek door Doll & Hill (1954) bleek
%   ...'')

%---------- Methodologie ------------------------------------------------------
\section{Methodologie}
\label{sec:methodologie}
Eerst zal er onderzocht worden welke combinatie's van narrow ai's een mogelijkheid zijn om verschillende aspecten van een menselijke conversatie ervaring na te bootsen.

Nadien zal er een  enquete gehouden worden om te bevragen waar voor hen de belangrijkheid van elk van deze onderzochte aspecten ligt.

De populatie zal opgesplitst worden in verschillende groepen:

\begin{itemize}
    \item Op vlak van AI kennis
    \item Op vlak van leeftijd
\end{itemize}

%---------- Verwachte resultaten ----------------------------------------------
\section{Verwachte resultaten}
\label{sec:verwachte_resultaten}

Er wordt verwacht dat men een beeld kan schetsen van welke narrow ai's men dient te combineren om verschillende menselijke aspecten van een conversatie ervaring na te bootsen.

Er wordt verwacht dat uit de enquete een duidelijk beeld kan gevormd worden wat voor de verschillende doelgroepen het gewicht van elk van deze aspecten is. Zodanig dat een developer die deze case zou willen uitwerken een idee heeft welke aspecten het belangrijkste zijn.

Ook verwacht men dat elk van de bevraagde aspecten ook effectief belangrijk zijn voor de populatie.

%---------- Verwachte conclusies ----------------------------------------------
\section{Verwachte conclusies}
\label{sec:verwachte_conclusies}

Er wordt verwacht dat men kan concluderen welke verschillende Artificial Narrow Intelligence zouden moeten samenwerken om een impressie te geven dat men een menselijke conversatie ervaring heeft met een software. Alsook verwacht men te kunnen concluderen waar voor de verschillende doelgroepen de belangrijkheid aan gehecht wordt.




%------------------------------------------------------------------------------
% Referentielijst
%------------------------------------------------------------------------------
% TODO: de gerefereerde werken moeten in BibTeX-bestand ``voorstel.bib''
% voorkomen. Gebruik JabRef om je bibliografie bij te houden en vergeet niet
% om compatibiliteit met Biber/BibLaTeX aan te zetten (File > Switch to
% BibLaTeX mode)

\phantomsection
\printbibliography[heading=bibintoc]

\end{document}
