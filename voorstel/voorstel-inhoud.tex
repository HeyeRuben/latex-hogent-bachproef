%---------- Inleiding ---------------------------------------------------------

\section{Introductie} % The \section*{} command stops section numbering
\label{sec:introductie}

Iedereen kent wel de term AI. Maar niet iedereen weet dat men dan spreekt over een Artificial Narrow Intelligence. Namelijk een AI die nog steeds beperkt is in wat het kan doen binnen een bepaalde context. Een narrow AI is beperkt tot het doen waarvoor het gemaakt is. Zo kan je bijvoorbeeld aan een helpdesk chatbox geen vragen stellen over een ander bedrijf of een ander domein dan waar die bot voor gemaakt is. \linebreak

Daar in tegen bestaat er ook Artificial General Intelligence. En dit is eigenlijk een vorm van AI dat dit alles wel kan. Een AI met de capaciteit om een verscheidenheid aan complexe problemen op te lossen in verschillende domeinen. Dat zichzelf autonoom controleert, met zijn eigen gedachten, zorgen, gevoelens, sterktes, zwaktes en neigingen. \linebreak

Sinds de opkomst van AI is Artificial General Intelligence eigenlijk het oorspronkelijke focus veld van AI. Maar al snel werd duidelijk dat het een zeer complex idee is. Sommigen denken dat het zelfs onmogelijk is om een AGI te bouwen. Terwijl mensen die meer ervaren zijn in het vak het slechts een technisch probleem vinden. \linebreak

Zeker met de opkomst van AI's zoals ChatGPT door OpenAI, die al in een zeer groot aantal domeinen kan antwoorden op vragen, is het interessant om te onzoeken naar wat juist een AGI is en waarom deze zo moeilijk te bouwen is. \linebreak

In deze bachelorproef zal er een ANI opgezet worden en zal er onderzocht worden wat de mogelijkheden zijn om deze verder te ontwikkelen naar een AGI. Hier zal men vooral focussen op onderstaande zaken: \linebreak

\begin{itemize}
    \item waarin verschilt een AGI van een ANI
    \item welke valkuilen zijn er bij het omvormen van een ANI naar een AGI
    \item waarom is het al dan niet mogelijk om een AGI te ontwikkelen
    \item zijn er al pogingen gedaan? En hoe liepen deze?
\end{itemize}

%---------- Stand van zaken ---------------------------------------------------

\section{State-of-the-art}
\label{sec:state-of-the-art}

Hier beschrijf je de \emph{state-of-the-art} rondom je gekozen onderzoeksdomein. Dit kan bijvoorbeeld een literatuurstudie zijn. Je mag de titel van deze sectie ook aanpassen (literatuurstudie, stand van zaken, enz.). Zijn er al gelijkaardige onderzoeken gevoerd? Wat concluderen ze? Wat is het verschil met jouw onderzoek? Wat is de relevantie met jouw onderzoek?

% Voor literatuurverwijzingen zijn er twee belangrijke commando's:
% \autocite{KEY} => (Auteur, jaartal) Gebruik dit als de naam van de auteur
%   geen onderdeel is van de zin.
% \textcite{KEY} => Auteur (jaartal)  Gebruik dit als de auteursnaam wel een
%   functie heeft in de zin (bv. ``Uit onderzoek door Doll & Hill (1954) bleek
%   ...'')

Je mag gerust gebruik maken van subsecties in dit onderdeel.

%---------- Methodologie ------------------------------------------------------
\section{Methodologie}
\label{sec:methodologie}

Hier beschrijf je hoe je van plan bent het onderzoek te voeren. Welke onderzoekstechniek ga je toepassen om elk van je onderzoeksvragen te beantwoorden? Gebruik je hiervoor experimenten, vragenlijsten, simulaties? Je beschrijft ook al welke tools je denkt hiervoor te gebruiken of te ontwikkelen.

%---------- Verwachte resultaten ----------------------------------------------
\section{Verwachte resultaten}
\label{sec:verwachte_resultaten}

Hier beschrijf je welke resultaten je verwacht. Als je metingen en simulaties uitvoert, kan je hier al mock-ups maken van de grafieken samen met de verwachte conclusies. Benoem zeker al je assen en de stukken van de grafiek die je gaat gebruiken. Dit zorgt ervoor dat je concreet weet hoe je je data gaat moeten structureren.

%---------- Verwachte conclusies ----------------------------------------------
\section{Verwachte conclusies}
\label{sec:verwachte_conclusies}

Hier beschrijf je wat je verwacht uit je onderzoek, met de motivatie waarom. Het is \textbf{niet} erg indien uit je onderzoek andere resultaten en conclusies vloeien dan dat je hier beschrijft: het is dan juist interessant om te onderzoeken waarom jouw hypothesen niet overeenkomen met de resultaten.

