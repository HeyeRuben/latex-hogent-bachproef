%---------- Inleiding ---------------------------------------------------------

\section{Introductie} % The \section*{} command stops section numbering
\label{sec:introductie}

Iedereen kent wel de term AI. Maar niet iedereen weet dat men dan spreekt over een Artificial Narrow Intelligence. Namelijk een AI die nog steeds beperkt is in wat het kan doen binnen een bepaalde context. Een narrow AI is beperkt tot het doen waarvoor het gemaakt is. Zo kan je bijvoorbeeld aan een helpdesk chatbox geen vragen stellen over een ander bedrijf of een ander domein dan waar deze chatbot voor gemaakt is. \linebreak

Daarentegen bestaat er ook Artificial General Intelligence. En dit is eigenlijk een vorm van AI dat dit alles wel kan. Een AI met de capaciteit om een verscheidenheid aan complexe problemen op te lossen in verschillende domeinen. Dat zichzelf autonoom controleert, met zijn eigen gedachten, zorgen, gevoelens, sterktes, zwaktes en neigingen. \linebreak

Sinds de opkomst van AI is Artificial General Intelligence eigenlijk het oorspronkelijke focus veld van AI. Maar al snel werd duidelijk dat het een zeer complex idee is. Sommigen denken dat het zelfs onmogelijk is om een Artificial General Intelligence te bouwen. Terwijl mensen die meer ervaren zijn in het vak het slechts een technisch probleem vinden. \linebreak

Zeker met de opkomst van AI's zoals ChatGPT door OpenAI, dat een chatbot is die al in een zeer groot aantal domeinen kan antwoorden op vragen, is het interessant om te onzoeken naar wat juist een AGI is en waarom deze zo moeilijk te bouwen is. \linebreak

In deze bachelorproef zal er een Artificial Narrow Intelligence opgezet worden en zal er onderzocht worden wat de mogelijkheden zijn om deze verder te ontwikkelen naar een Artificial General Intelligence. Hier zal men vooral focussen op onderstaande zaken: \linebreak

\begin{itemize}
    \item waarin verschilt een AGI van een ANI
    \item welke valkuilen zijn er bij het omvormen van een ANI naar een AGI
    \item waarom is het al dan niet mogelijk om een AGI te ontwikkelen
    \item zijn er al pogingen gedaan? En hoe liepen deze?
\end{itemize}

%---------- Stand van zaken ---------------------------------------------------

\section{Literatuurstudie}
\label{sec:state-of-the-art}

In deze literatuurstudie zal er een beeld gevormd worden over wat Artificial Narrow Intelligence en Artificial General Intelligence juist is.

\subsection{Artificial Narrow Intelligence}

De term 'Artificial Intelligence' is voor het eerst benoemd in 1955 door John McCarthy. Sinds de eerste aanraking met AI is er ondertussen al enorm veel dat AI de dag van vandaag kan, en is er nog steeds veel ruimte voor uitbreiding.

Wat veel mensen niet weten is dat wanneer men refereert naar AI of Artificial Intelligence, dat men eigenlijk praat over Artificial Narrow Intelligence, waarom zal verder in deze literatuurstudie besproken worden. 

De grootste groei van AI de afgelopen jaren kan men vinden in enerzijds perceptie en anderzijds cognitie. In perceptie is de grootste evolutie gebeurd op vlak van spraakherkenning. Dit is iets dat door tal van mensen wordt gebruikt, denk bijvoorbeeld maar aan Alexa, Siri,... Hetzelfde geldt voor afbeeldingsherkenning.

Wetende dat bijvoorbeeld spraakherkenning pas sinds de zomer van 2016 enorm is beginnen groeien kan me wel stellen dat evolutie in AI enorm snel kan gaan, dit is ook de reden waarom in deze bachelorproef de evolutie van artificial narrow intelligence naar artificial general intelligence onderzocht zal worden, hier over later meer. \autocite{brynjolfsson2017artificial}

Als we gaan kijken naar wat Artifical Narrow Intelligence juist is, kan men stellen dat dit een AI is die gebouwt is om een bepaalde taak uit te voeren binnen een bepaald domein. Eventuele kennis die deze AI vergaart door deze taak uit te voeren zal niet automatisch worden gebruikt om andere taken uit te voeren buiten dit domein. 

In tegenstelling met Artificial General Intelligence, die complexe denkprocessen probeert na te bootsen, is Artificial Narrow Intelligence ontworpen om een specifieke taak zonder menselijke hulp tot een goed einde te brengen.  

\subsection{Artificial General Intelligence}

In tegenstelling tot Artificial Narrow Intelligence verwijst Artificial General Intelligence naar een AI die even veel kan als de cognitieve systemen van een mens. Hiermee wordt bedoelt dat die verschillende soorten taken, uit verschillende domeinen kan uitvoeren. Door het uitvoeren van deze taken leert de AI bij zoals een mens dit doet en kan hij bijleren om vervolgens taken uit andere domeinen ook te kunnen oplossen. 

Artificial General Intelligence heeft het vermogen om kennis te verwerven en toe te passen, en om te redeneren en te denken, op verschillende gebieden. Je kan stellen dat Artificial General Intelligence streeft naar 'Algemene Intelligentie', zoals dit ook uit de engelse term afgeleid kan worden. 

Uit een studie is gebleken dat wanneer we menselijke algemene intelligentie willen toepassen op een AI, die het volgende moet bevatten: \linebreak

\begin{itemize}
    \item Het vermogen om algemene problemen op een niet-domeingebonden manier op te lossen
    \item Het vermogen om specifieke (taakgebonden) en algemene intelligentie samen te gebruiken
    \item Het vermogen om te leren van zijn omgeving, andere ai's en leraren, al dan niet menselijk
    \item Het vermogen om beter te worden in het oplossen van nieuwe soorten problemen naarmate de ai meer ervaring op doet met dit type problemen
\end{itemize}

Het belangrijkste om te onthouden is dat een belangrijk aspect van Artificial General Intelligence draait om het feit dat een systeem autonoom kan leren, en kennis die het verstrekt heeft door een taak te voltooien kan gebruiken om een andere taak op te lossen, ongeacht welk domein deze taak zich in bevindt. 

Het systeem moet kunnen interageren met zijn omgeving en met andere entiteiten in zijn omgeving, dit kunnen ai's of menselijke interacties zijn. De AI leert bij uit deze interacties. Verder is het in staat om verder te bouwen op eerdere ervaringen en de vaardigheden die daar geleerd zijn te onthouden en toe te passen op complexere taken, om zo ook deze af te ronden. Op deze manier leert een Artificial General Intelligence als maar bij en kan het steeds complexere doelen bereiken.

\autocite{goertzel2007artificial}

% Voor literatuurverwijzingen zijn er twee belangrijke commando's:
% \autocite{KEY} => (Auteur, jaartal) Gebruik dit als de naam van de auteur
%   geen onderdeel is van de zin.
% \textcite{KEY} => Auteur (jaartal)  Gebruik dit als de auteursnaam wel een
%   functie heeft in de zin (bv. ``Uit onderzoek door Doll & Hill (1954) bleek
%   ...'')

%---------- Methodologie ------------------------------------------------------
\section{Methodologie}
\label{sec:methodologie}
Enerzijds zal het onderzoek er uit bestaan om een Artificial Narrow Intelligence op te zetten via het Azure platform, en er zal gekeken worden wat er nodig is om dit op te zetten.

Anderzijds zal er onderzocht worden wat er tyerend is aan Artificial General Intelligence en wat hier de valkuilen en opportuniteiten zijn. 

Er zal verder ook bekeken worden of het al dan niet mogelijk is om de opgezette Artificial Narrow Intelligene om te vormen naar een Artificial General Intelligence. 

%---------- Verwachte resultaten ----------------------------------------------
\section{Verwachte resultaten}
\label{sec:verwachte_resultaten}

Bij het opstellen van de Artificial Narrow Intelligence wordt er verwacht dat dit lukt en er zaken kunnen vastgesteld worden die hier typisch aan zijn, op deze manier kan men dit gebruiken om te vergelijken en te toetsen of het mogelijk is om deze om te vormen naar een Artificial General Intelligence.


%---------- Verwachte conclusies ----------------------------------------------
\section{Verwachte conclusies}
\label{sec:verwachte_conclusies}

Hier beschrijf je wat je verwacht uit je onderzoek, met de motivatie waarom. Het is \textbf{niet} erg indien uit je onderzoek andere resultaten en conclusies vloeien dan dat je hier beschrijft: het is dan juist interessant om te onderzoeken waarom jouw hypothesen niet overeenkomen met de resultaten.

Er wordt verwacht dat het opzetten van een Artificial Narrow Intelligence mogelijk zal zijn, en dat er een duidelijke analyse kan gemaakt worden van wat er essentieel is om dit op te zetten. 

Er wordt verwacht dat het onderzoeken wat er essentieel is om een Artificial General Intelligence op te zetten complex zal zijn, maar er hier zeker een duidelijk beeld over kan gevormd worden. Het effectief opzetten (en laten werken) van een Artificial General Intelligence wordt verwacht niet te lukken, wel zal er veel geleerd worden over waarom dit juist niet lukt.

